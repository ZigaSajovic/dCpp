\documentclass{article}

\usepackage[slovene]{babel}
\usepackage[utf8]{inputenc}
\usepackage{amsmath}
\usepackage{amsfonts}
\usepackage{amsthm}
\newcommand{\RR}{\mathbb{R}}
\newcommand{\II}{\mathbb{I}}
\newcommand{\JJ}{\mathbb{J}}
\newcommand{\E}{\mathcal{E}}
\newcommand{\T}{\mathcal{T}}
\newcommand{\VV}{\mathcal{V}}
\newcommand{\MM}{\mathcal{M}}
\newcommand{\NN}{\mathcal{N}}
\newcommand{\F}{\mathcal{F}}
\newcommand{\D}{\mathcal{D}}
\newcommand{\Op}{\partial^{\bigoplus}}
\newcommand{\op}[1]{\partial^{#1\bigoplus}}
\DeclareMathOperator{\interior}{int}
\newtheorem{definicija}{Definicija}[section]
\newtheorem{izrek}{Izrek}[section]
\title{Odvedljivi stroji}
\author{Žiga Sajovic, Martin Vuk}
\begin{document}
\maketitle
\begin{abstract}
  Dandanes matematične funkcije pogosto računamo z računalniki. Če je funkcija
  odvedljiva, nas včasih zanima tudi, koliko je njen odvod. Večina bralcev verjetno
  pozna simbolično odvajanje izrazov ali numerično odvajanje s končnimi
  diferencami. V tem prispevku bova predstavila še en način, kako lahko določimo
  odvode funkcije, ki jo računamo z računalniškim programom. \emph{Avtomatsko
    odvajanje} je skupno ime za različne postopke, s katerimi program, ki računa
  vrednosti odvedljive funkcije, preoblikujemo v program, ki izračuna odvod. 
  Večina algoritmov za avtomatsko odvajanje znajo to narediti brez dodatnega
  človekovega posega (odtod ime avtomatski). Za razliko
  od numeričnega odvajanja ne trpijo za izgubo natančnosti. Prednost pred
  simboličnim odvajanjem, pa je v tem, da jih lahko uporabimo tudi v primeru, ko
  ne poznamo eksplicitne formule, ampak vrednosti računamo s programom.

  Avtomatsko odvajanje je uporabno povsod, kjer potrebujemo 
  odvode funkcij, ki jih računamo s kompleksnimi algoritmi. Tipični primeri so
  optimizacija parametrov pri strojnem učenju, računanje odvoda preslikave, ki nima
  eksplicitne formulacije (npr. Poincarejeva preslikava za dinamične sisteme,
  lastne vrednosti matrik, ...).
\end{abstract}
\section{Uvod}
Za začetek si oglejmo preprost primer, na katerem ilustriramo osnovno idejo.
Recimo, da računamo vrednosti funkcije $y=\sqrt{x}$ z babilonskim oziroma Haronovim obrazcem.
Pri tem postopku, najprej izberemo začetni približek za
kvadratni koren $y_0=y_z$. Nato pa z rekurzivno formulo
\begin{equation*}
  y_n=\frac{1}{2}\left( y_{n-1}+\frac{x}{y_{n-1}} \right)
\end{equation*}
računamo člene zaporedja $y_n$. Zaporedje $y_n$ konvergira zelo hitro k
vrednosti $\sqrt{x}$, zato so vrednosti $y_n$ že po nekaj korakih zelo blizu
iskani vrednosti $\sqrt{x}$. Opisani postopek lahko prevedemo v računalniški
program. Na primer v programskem jeziku Python, bi program izgledal takole 

\begin{verbatim}
y = x/2
while abs(y*y-x) > 5e-11:
  y = (y+x/y)/2
\end{verbatim}

Pri tem smo za začetni približek izbrali $x/2$, lahko pa bi izbrali tudi kaj
drugega. Vrednost pa nas zanima na 10 decimalk natančno. 
Za funkcijo $f(x)$ želimo izračunati tudi njen odvod $f'(x)$. V našem
primeru lahko odvod preprosto izrazimo s funkcijo $f'(x)=\frac{1}{2\sqrt{x}}$,
ampak zamislimo si, da funkcije $f$ sploh ne poznamo, ampak poznamo le računalniški program za
njen izračun. Še vedno pa bi radi izračunali vrednosti odvoda $f'(x)$. Uporabimo
lahko numerične približke za odvod, vendar pri tem izgubimo natančnost. 

Osnovna ideja avtomatskega odvajanja je v tem, da odvajamo program. Računalniški program ni nič drugega kot zaporedje osnovnih
računskih operacij in premetavanja vrednosti po pomnilniku. V našem primeru
uporabljamo dve spremenljivki \texttt{x} in \texttt{y}, zato si lahko vsako vrstico
programa, ki spreminja vrednosti \texttt{x} ali \texttt{y} predstavljamo kot
funkcijo $\phi_i:\RR^2\to\RR^2$. Označimo z $(x_k,y_k)$ vrednosti spremenljivk
\texttt{x} in \texttt{y}, po izvedbi $k$-te vrstice programa in pred izvedbo
$k+1$-ve vrstice. Vsaka vrstica programa določa preslikavo
\begin{eqnarray*}
  \phi_k&:&\RR^2\to \RR^2\\
  \phi_k&:&(x_{k-1},y_{k-1})\mapsto (x_k,y_k).
\end{eqnarray*}
Celoten program predstavlja preslikavo $F:(x_0,y_0)\mapsto (x_n,y_n)$, ki jo
lahko predstavimo kot kompozitum preslikav $\phi_k$, defniranih z vsako vrstico programa:
\begin{equation*}
  F(x_0,y_0) = \phi_n\circ\phi_{n-1}\circ \ldots \phi_1(x_0,y_0) = \phi_n(\phi_{n-1}(\ldots \phi_1(x_0,y_0)\ldots))
\end{equation*}

Odvod preslikave $F$ lahko po verižnem pravilu izrazimo kot produkt Jacobijevih matrik
$D\phi_i$ preslikav $\phi_i$. Ker nas zanima le odvod $y$ po $x$, matriko $DF$
pomnožimo z leve in desne z vektorjem $(0,1)$ 
\begin{equation}
\label{eq:kompozitum}
  \frac{dy}{dx} =
  \begin{bmatrix}
    0& 1
  \end{bmatrix}
\cdot D\phi_n(x_{n-1},y_{n-1})\cdot D\phi_{n-1}(x_{n-2},y_{n-2})\ldots D\phi_1(x_0,y_0)\cdot
  \begin{bmatrix}
    0\\
    1
  \end{bmatrix} 
\end{equation}
Program za računanje odvoda lahko povsem sledi originalnemu programu, le da si
mora na vsakem koraku zapomniti vmesne vrednosti odvoda. Na vsakem
koraku programa, si mora program za odvod zapomniti vektor
\begin{equation*}
\begin{bmatrix}dx_{k}\\dy_{k}\end{bmatrix} = D\phi_{k}(x_{k-1},y_{k-1})\ldots D\phi_1(x_0,y_0)\cdot \begin{bmatrix} 0\\ 1\end{bmatrix} 
\end{equation*}
Vrednost $(dx_k,dy_k)$ lahko definiramo rekurzivno
\begin{equation}
  \label{eq:rek_odvod}
\begin{bmatrix}dx_{k}\\dy_{k}\end{bmatrix} = D\phi_{k}(x_{k-1},y_{k-1})\cdot \begin{bmatrix} dx_{k-1}\\ dy_{k-1}\end{bmatrix} 
\end{equation}
V našem primeru je 
\begin{equation*}
  \phi_1(x,y)=(x,x/2)\text{ in } \phi_n(x,y)=\left(x,\frac{1}{2}\left( y+\frac{x}{y} \right)\right)
\end{equation*}
Vrednost spremenljivke $x$ se ne spreminja tekom programa, zato je dovolj, da
spremljamo le vrednosti odvoda po $y$
\begin{equation*}
  \frac{d}{dx}\phi_0(x,y(x))_2=1/2\text{ in }\frac{d}{dx}\phi_k(x,y(x))_2=\frac{1}{2}\left( \frac{dy}{dx}+\left(\frac{1}{y}-\frac{x}{y^2}\frac{dy}{dx}\right) \right)
\end{equation*}
 
Naslednji program poleg vrednosti funkcije (spremenljivka \texttt{y}) računa
tudi njen odvod (spremenljivka \texttt{dy})
\begin{verbatim}
dy = 0.5
y = x/2
while abs(y*y-x) > eps:
  dy = (dy + (1/y - x/(y*y)*dy)/2
  y = (y+x/y)/2
\end{verbatim}

\section{Avtomatsko odvajanje}
Na uvodnem primeru smo videli, da si lahko računalniški program predstavljamo
kot vektorsko funkcijo več spremenljivk. Vektorska funkcija, ki predstavlja
program, je podana kot kompozitum preslikav, ki ustrezajo posameznim vrsticam
programa. Program za odvod smo napisali tako, da smo vsaki vrstici
programa dodali vrstico, ki je izračunala odvod. Ta postopek lahko namesto nas
opravi računalnik (od tod ime avtomatski odvod). V nadaljevanju bomo opisali dva
načina, kako je to izvedeno v obstoječih knjižnicah. 

Oglejmo si matematični model, s katerim lahko utemeljimo postopke avtomatičnega
odvajanja. V našem modelu bomo zajeli le spremenljivke predstavljene s števili s
plavajočo vejico,  s katerimi v računalniku predstavljamo realna števila. 
Ostale spremenljivke kot so števci v zankah, celoštevilske vrednosti, logične
vrednosti in podobne ne bodo vključene v našem modelu, ampak jih upoštevamo
zgolj kot parametre, ki sicer vplivajo na potek programa, vendar njihovega
vpliva ne moremo infinitezimalno obravnavati. Če predpostavimo, da so vse
spremenjivke, ki nas zanimajo tipa \texttt{float} (ali \texttt{double)}), si lahko stanje spomina
predstavljamo kot $n$-razsežni realni vektor\footnote[1]{omejitev, da je
  spremenljivka tipa \texttt{float}, smo vpeljali zgolj zaradi enostavnosti.
  Odvajamo lahko po poljubnem tipu, ki to dopušča. Če bi npr. definirali
  poseben tip, ki bi predstavljal funkcijo, bi na isti način lahko računali
  funkcijski odvod, kot ga poznamo v variacijskem računu.}. Vsaka spremenljivka(lokacija v
spominu) predstavlja eno komponento tega vektorja. Množico vseh možnih stanj
spomina, s katerim razpolaga program, lahko modeliramo z $n$-razsežnim
vektorskim prostorom $V\simeq\RR^n$. Vektorski prostor $V$ bomo imenovali
\emph{virtualni pomnilnik programa}. Računalniški program, ki
ima na voljo $n$-mest v virtualnem pomnilniku, si lahko predstavljamo kot preslikavo
\begin{equation}
  \label{eq:program_kot_preslikava}
  P: V\to V,
\end{equation}
kjer je $V$ vektorski prostor dimenzije $n$. Množica vseh takih preslikav
opremljena z operacijo kompozituma je \emph{monoid}.
\begin{definicija}
  Naj bo $V$ Banachov vektroski prostor. Preslikava $P:V\to V$ je odvedljiva v
  točki $x\in V$, če obstaja
  linearen omejen operator $TP_x:V\to V$, za katerega je
  \begin{equation}
    \label{eq:frechet}
    \lim_{h\to 0}\frac{\|P(x+h)-P(x)-TP_x(h)\|}{\|h\|} = 0.
  \end{equation}
  Preslikavo $TP_x$ imenujemo \emph{Fréchetov odvod} ali \emph{linearizacija}
  preslikave $P$.
\end{definicija}
Za preslikave $\RR^n\to \RR^m$ lahko Fréchetov odvod
izrazimo kot množenje vektorja $h$ z Jacobijevo matriko parcialnih odvodov
komponent preslikave $P$
\begin{equation*}
  TP_x(h) = JP(x)\cdot h.
\end{equation*}
Preslikave, ki ustrezajo osnovnim ukazom in operacijam v določenem
programskem jeziku, predstavljajo generatorje vseh možnih programomv v
monoidu. Označimo z $\E$ množico vseh elementarnih ukazov in operacij, ki so
implementirane v programskem jeziku. Označimo z $\T$ prostor končnih programov,
ki so generirani z elementarnimi operacijami iz $\E$
\begin{equation*}
  \T=\langle E \rangle
\end{equation*}

Predpostavimo za trenutek, da so vse elementarne operacije povsod odvedljive. 

\begin{izrek}
  Naj bo $\tau:V\times \T\to \T$, $\tau:(x,P)\mapsto TP_x$ preslikava,  ki vsakemu programu
  $P$ priredi njegov odvod v dani točki $x$. Predpostavimo še, da sta med generatorji $\E$
  tudi operaciji seštevanja in množenja. Za vsak $x\in V$ je preslikava
  $\tau_x:P\mapsto \tau(x,P)$ avtomorfizem monoida $(\T,\circ)$. 
\end{izrek}
Če $\E$ vsebuje seštevanje in množenje, potem so v $\T$ tudi vse linearne preslikave
na $V$. Tako preslikava $\tau_x$ res slika nazaj v $\T$ Dejstvo, da je $\tau_x$ homomorfizem za operacijo kompozituma, je pa 
neposredna posledica verižnega pravila.

\subsection{Odvedlji stroji}
V nadaljevanju bomo definirali, kako si predstavljamo odvedljiv programski jezik
oziroma računski stroj. Odvedljiv stroj si bomo predstavljali kot računski
stroj, ki izvaja zaporedne enostavne operacije, ki so odvedljive, na virtualnem
pomnilniku $V\simeq\RR^n$. Prostor $\T$ opisuje vse možne programe, na našem
stroju, množica generatorjev $\E$ pa elementarne operacije, ki jih stroj lahko
izvaja. Ker je $\T=\langle \E \rangle$, lahko vsak program $P$ zapišemo kot
kompozitum elementarnih operacij $\epsilon_i$:
\[
P = \epsilon_k\circ\epsilon_{k-1}\ldots\epsilon_0.
\]  
\begin{definicija}
 Banachov računski stroj je trojica $(\T,V,)$  
\end{definicija}
Če želimo tekom programa sproti računati tudi odvode, moramo hraniti vmesne
vrednosti odvodov. Zato moramo pomnilnik (vektorski prostor $V$) razširiti, da
bo vseboval tudi odvode. Odvod je linearna preslikava $V\to V$, prostor
vseh linearnih preslikav $\mathcal{L}(V)$ je izomorfen tenzorskemu produktu
$V^*\otimes V$. Tako za izvedbo programa, ki računa odvode, potrebujemo virtualni
pomnilnik izomorfen prostoru $V\oplus (V^*\otimes V)$. Poleg vsake spremenljivke $x_j$, ki
jo uporabimo v programu, si moramo zapomniti še vrednosti odvodov po vseh
ostalih spremenljivkah $\frac{\partial x_j}{\partial x_i}$.   

Za višje odvode rekurzivno definiramo zaporedje vektorskih prostorov $V_n$, ki
predstavljajo virtualni pomnilnik
\begin{eqnarray}\label{eq:V_n}
  \label{eq:prapor}
  V_0 &=& V\\
  V_{n+1}&=&V\oplus (V^*\otimes V_n),
\end{eqnarray} 
ki ga potrebuje program za izračun vrednosti $n$-tih odvodov. Izbira $V_n\simeq
V\oplus (V^*\otimes V) \oplus (V^*\otimes V^*\otimes V) \ldots \oplus
(V^*)^{\otimes n}\otimes V$ odraža dejstvo, da za izračun vrednosti $n$-tega odvoda v neki točki $x\in V$
potrebujemo vrednosti funkcije in vseh nižjih odvodov.
\begin{definicija}
  Direktno limito zaporedja vektorskih prostorov $V_n$ imenujemo
  \emph{univerzalni virtualni prostor} in jo označimo z $V_\infty$.
\end{definicija}
\begin{izrek}
  Univerzalni virtualni prostor je izomorfen tenzorskemu produktu virtualnega
  prostora $V$ in tenzorske algebre $T(V^*)$ duala virtualnega prostora $V^*$:
  \begin{equation}
\label{eq:tenzor_algebra}
    V_\infty\simeq V\otimes T(V^*) = V \otimes\left(\bigoplus_{k=0}^\infty (V^*)^{\otimes k} \right)
\end{equation}

\end{izrek}

Elementi $V_\infty$ predstavljajo neskončne potenčne vrste kot preslikave 
$V\to V$. Prostor $(V^*)^{\times n}\otimes V$ je izomorfen prostoru
multilinearnih preslikav $V^{\times n}\to V$. Če poljubno
multilinearno preslikavo komponiramo z vložitvijo $V\to V^{\times n}$ na
diagonalo, dobimo preslikavo $V\to V$, ki se v koordinatah izraža kot linearna
kombinacija monomov stopnje $n$.
   
Naj bo $\mathrm{P}\subset\T$ preslikava $V_\JJ\to V_\II$. Slika predstavlja
mnogoterost $\mathcal{M}\subset V_\II$ (dejansko je unija zlepkov kot definrano
v \ref{eq:zlepek}).


Spomnimo se popolnega diferenciala preslikave
$$df=\sum_{\forall_j}\frac{\partial f}{\partial x_j}dx_j, x_i\in V_{j\in\JJ}$$

%tukaj morda malo o dualnih številih, morda tisto spodaj E->E+dE ni dovolj za povprečnega bralca. Čeprav je ta definicja, skozi tenzorski produkt in pushforward enostavnejša (vsaj zame) in čisto splošna, omogoča izračune poljubnega števila hkratnih odvodov, pri dualnih številih moraš vedno nova uvajat, in izpeljevat nova pravila za vsako dodano število. Morda je ravno to tukaj na mestu, da dodava kaj je pomankljivost tistega sistema, in kako jih najin pristop reši.
%http://adl.stanford.edu/hyperdual/Fike_AD2012_slides.pdf
%tukaj izpeljujejo dualna za več odvodov hkrati in za 2 red. Vidiš kako se že komplicira, za 2 spremenljivki in 2. red? Najina formulacije čist vse o pokrije z enovito definicijo, ne rabiva skos "štukat"

in naj $dx_j$ predstavljajo bazo duala $V_\JJ^{*}$. Poslužili se bomo tako povleka (Obratno), kot potiska (Direktno), odvisno od implementacije.
\begin{equation}\label{eq:pushfirward}
 d_p\T: T_pV\to T_{\T x}V
\end{equation}
\begin{equation}\label{eq:pullback}
 d_p\T(\frac{\partial}{\partial u^a})=\frac{\partial\T^b}{\partial u^a}\frac{\partial}{\partial v^b}
\end{equation}
Potem lahko definiramo prostor, ki ga generiramo kot direktno vsoto $V\bigoplus (V^*\bigotimes V)$.
  
  Definirajmo operator 
  \begin{equation}
  \D=1+d
  \end{equation}
  \begin{equation}\label{eq:dirSumFun}
  \D:\T\to\T\oplus d\T
  \end{equation}
  Potem
  \begin{equation}\label{eq:dirSumV}
    \T\oplus d\T: V\oplus (V^*\otimes V)\to V\oplus (V^*\otimes V)
    \end{equation}
  
  Skupaj z definicijami prejšnjih sestavkov, lahko konstruiramo odvedljiv abstrakten stroj, ki z opisano mehaniko razširja množico Turing-izračunljivih strojev.
  
  \begin{equation}\label{eq:dTuring}
  M=\langle\T, V, d\rangle
  \end{equation}

 \begin{itemize}
 \item
 $V\subset V_\infty$ služi kot množica stanj, domena ter "neskončen trak"
 \item
 $\T$ je monoid operacij nad $V$
 \item
 $d$ je diferencialni operator
% \item
% $\mathcal{K}$ kontrolne strukture
 \end{itemize}
 %na te mestu dodati navodila konstrukcije odvodov višjega reda, skozi rekurzivno definicjo, kjer dualu prišemo svoj dual. Morda pa je bolje dativ poglavje zase
 
 Karkoli operira pod temi specifikacijami, je odvedljiv Turingov stroj.
 
 Ob konstrukciji $M$, je potrebna le enkratna aplikacija $\Op$. Naj $\T$ označuje programski jezik generiran z $\langle\E\rangle$ in $M$ stroj, ki ga implementira. Definirajmo stroj prvega reda
 \begin{equation}\label{eq:Mprime}
 %\Op V je prostor v katerem se dogaja, \Op\T so preslikave/operatorji nad tem prostorom
	 \D M=\langle\D\T, V\oplus (V^*\otimes V), d\rangle
 \end{equation}
 oziroma v poznani, konceptualno ekvivalentni notaciji
  \begin{equation}\label{eq:M'}
 	 M'=\langle\T', V_1, d\rangle
  \end{equation}
  
  Tako je operator $\op{}$ preslikava stroja $M$ v odvedljiv stroj $M'$. Velja 
  \begin{equation*}
	M\subset M'
  \end{equation*} 
  
  \begin{izrek}\label{izr:dM}
	$M'=\langle\T', V_1, d\rangle$ je odvedljiv računski stroj.
  \end{izrek}
  Dokaz je trivialen, raširjen operator $\circ$ je dobro definiran, tako je $(\T', \circ)$ monoid in $V_1\subset V_\infty$.
  
  Vsi programi konstruirani v $M'$, skozi zaporedje akcij $\T'$ nad $V_1$, v vsaki točki izvajanja vsebujejo informacijo spremembe prvega reda. Tak program označimo s $P'$. Potem je $P'V_1$ mnogoterost, na vsaki točki razširjena s kotangentnim svežnjem.
 
 Z ustrezno izbiro projekcije $\mathcal{P}_\II$ programa, kot $\mathcal{M}\subset V$, omogočimo preučevanje njegovega učinka na $V$. Projeciramo lahko na enotske vektorje, ki predstavljajo konceptualni vhod programa in proučujemo spremembe akcije nad $V$. S tem merimo odzivnost programa na vhodne podatke. Diagram izvajanja programa lahko predstavimo z usmerjenim grafom;- potem lahko projeciramo na, in odvajamo po kateremkoli izmed njegvih oglišč.
  
 \subsection{Višji odvodi}
 
 V nadaljnjem se ob odobravanju n-tega reda predpostavlja $\E\subset\mathcal{C}^n$, kar implicira obstoj $\partial^n\T$. Potenciranje operatorja $d$ označimo z
\begin{equation}\label{eq:d}
d^nf=\sum_{\forall_{i,..,k}}\frac{\partial^n f}{\partial x_i...\partial x_k}dx_i\otimes dx_j\otimes...\otimes dx_k
\end{equation}
Slika $d^nf$ je element $V\otimes(V^*)^{n\otimes}$.
Potem lahko definiramo
\begin{equation}\label{eq:op}
\D^n=1+ d+ d^2+...+ d^n
\end{equation}

\begin{equation}
	\D^n=\sum\limits_{n=0}^{n}\sum_{\forall_{i,..,k}}\frac{\partial^n }{\partial x_i...\partial x_k}dx_i\otimes dx_j\otimes...\otimes dx_k
\end{equation}
 in slika $\E\to\sum\limits_{i=0}^nd^i\E$. Rekurzivna defincija odraža strukturo prostora $\subseteq_{\forall_i}\D^i\T$. Algebra je dobro definirana. Izračun pokaže
 \begin{equation}
 \D^n=\frac{1- d^{n+1}}{1- d}
 \end{equation}
 Tako so preslikave, ki operirajo nad univerzalnim vektorskim prostorom $V_n$ definirane kot slike operatorja $\D^n$. Operator apliciramo na $\T$.
 \begin{equation}
 	\D^n:\T\to \T^{n'}
 \end{equation}
 \begin{equation}\label{eq:t^n}
 	\T^{n'}:V_m\to V_n 
 \end{equation}
 za $m\le n$. Gre poudarit, da je mogoče operaciji povleka $\eqref{eq:pullback}$ in potiska $\eqref{eq:pushfirward}$ izvesti le do $m$-tega reda, preostali pa vsebujejo le (v izvedbi) lokalno informacijo. V praksi uporabimo $m=1$, za (v izvedbi) lokalen gradient in $m=n$ za poln gradient. \label{note:m<=n}
 
 Preslikave $\T^{n'}$ so podmnožica splošnejšega monoida $\F^n=\{f:V_m\to V_n\}$. Programe generirane s $\T^{n'}$, označimo s $P^{n'}$.
 
 $$P^{n'} = \epsilon^{n'}_k\circ\epsilon^{n'}_{k-1}\ldots\epsilon^{n'}_0.$$
 \begin{izrek}
 %Morda morava tukaj govorit o \Op{n} V
 Dimenzija prostora $V_n$ je $\frac{1-k^{n+1}}{1-k}$, kjer je k dimenzija $V$.
 \end{izrek}
 
 Dokaz je trivialen. Spominska kompleksnost modela je očitno eksponentna. Gre poudarit, da v praksi redko projeciramo na celoten $V$, pogosteje nas zanima manjša podmnožica komponent. Ob interpretacija programa skozi usmerjen graf, to pomeni, da merimo akcijo specifični oglišč v njem (ki eliminirajo vsa ostala oglišča, iz katerih obstaja pot do oglišča na katerega že projeciramo).
 
 %morda tukaj nasloviva probleme z višjimi odvodi v drugih sistemov, referencirava tista 2 autorja, in da tole elegantno reši problem. Dotično, ona najavata potrebo po odvajanju programov, ki v svojem poteku izračunavajo odvode. Recimo delaš specifično numerično analizo, in v nekih korakih rabiš gradient F, in posledično je tvoj program odvisen od teh odvodov. V najini analizi je to enostavno rešeno, saj imava prostor, kjer imajo vsi odvodi koordinate, nižji so vsebovani v prostoru višjih. To odraža dejstvo da lahko katerokoli vrednost v pomnilniku ki ti pripada, uporabiš v računih.
 
 \subsection{$\T$-calculus}
 
 V tej fomrulaciji je stroj $M$ funkcijski prostor,  na katerega elemente lahko apliciramo metode funkcionalne analize. Začnimo s preprostejšimi.
 
 Analizo motiviramo s primerom. Predpostavimo časovno kompleksen algoritem $A$, za katerega iščemo približek. Z opisano mehaniko lahko program lineariziramo.
 Izberemo $v\in PV$ in preslikamo v $v'\in P'V_1$ (kar je potrebno storiti le enkrat). Potem lahko program $PV$ približamo z linearno aproksimacijo $v+(v'-v) (V)$, ki je linearne kompleksnost $\mathcal{O}(k)$, kjer je $k$ dimenzija $V$.
 %enačba pomeni, da vektorju odštšteješ "realni" del
 Slika linearizacije služi kot dober približek slike prvotnega programa.
 
 
 Za nadaljevanje je prikladno vnesti nekaj notacije. Naj tenzor $\VV^{ik}_{\JJ_k}\in V_n (\ref{eq:V_n})$ predstavlja element virtualnega prostora. $\JJ_k$ je nabor indexov, $\lvert\JJ_k\rvert$ je enak redu odvoda. Index $k=\lvert\JJ_k\rvert$, zato ga v zapisu izpuščamo. 
 \begin{equation}\label{eq:tenV}
 	\VV^{ik}_{\JJ_k}= \bigoplus\limits_{0\le k\le n} \VV^{ik}_{\JJ_k}
 \end{equation}
 Tako je $\eqref{eq:tenV}$ preslikava iz tenzorske algebre $T(V)$ v $V$
 \begin{equation}\label{eq:vto}
 	\VV^{ik}_{\JJ_k}:\bigoplus\limits_{0\le k\le n}V^{\JJ_k}_{k}\to V^i
 \end{equation}
 V tej notaciji linearizacija postane
 \begin{equation}\label{eq:linear}
	\VV^i+\VV^i_j(v^j-v_0^j)
 \end{equation}
 
 Induktivno bi želeli program razvit v vrsto. Poslužimo se enakosti
\begin{equation}
	e^{hd}=\sum\limits_{i=0}^{\infty}\frac{(hd)^i}{i!}
\end{equation}
kjer je $d^i$ $\eqref{eq:d}$. Zato je
\begin{equation}\label{eq:e^d}
	e^d=\sum\limits_{n=0}^{\infty}\frac{1}{n!}\sum_{\forall_{i,..,k}}\frac{\partial^n }{\partial x_i...\partial x_k}dx_i\otimes dx_j\otimes...\otimes dx_k
\end{equation}
Izraz $\eqref{eq:e^d}$ je operator
\begin{equation}
	e^d:\F\to\F^n
\end{equation}
Za specifičen program $P\in\T$ velja
\begin{equation}
	(e^d)(P):V^i\to \VV^{ik}_{\JJ_k}\in V_n^i
\end{equation}
saj so gre za zaključen izraz; če bi nad izrazom dalje izvajali potisk/povlek, bi slikali $V^i_n\to\VV^{ik}_{\JJ_k}\in V^i_n$ $\eqref{eq:t^n}$.

 Po $\eqref{eq:vto}$ je slika element komutativno asociativne unitalne algebre $S(V^{i*})$, kar korespondira polinomom z nedoločenkami v $V^i$. Tako je razvoj v vrsto moč izrazit kot srkčitev
 \begin{equation}\label{eq:tenzorVrsta}
 	P(V^i+v) = \Big((e^d)(P)(V^i)\Big)v^{\JJ_k}_{k}
 \end{equation}
 
 $$P(V^i+v) = \VV^{ik}_{\JJ_k}v^{\JJ_k}_{k}$$
 
ki je upravičena z opažanjem, da za vsak $n$ in vse nabore indeksov velja
$$\lim\limits_{\lVert v\rVert\to 0}\frac{\partial^n }{\partial v_i...\partial v_k}\text{(leva stran)}=\lim\limits_{\lVert v\rVert\to 0}\frac{\partial^n }{\partial v_i...\partial v_k}\text{(desna stran)}$$
kjer se razume, da je $v^{\JJ_k}_{k}$ generiran z enim $v\in V^i$.

Slika skrčitve je element prvotnega virtualnega prostora $V^i$. Neodvisnost $(\ref{eq:tenzorVrsta})$ od koordinatnega sistema, se v programiranju prevede na neodvisnost v izvajanju. Tako je izraz $(\ref{eq:tenzorVrsta})$ invarianten stopnji izvajanja programa (upoštevajoč $\eqref{eq:t^n}$). 

\subsection{Optimizacija prostorske zahtevnosti}
Ponavadi nas ne zanima odvod po vseh spremenljvkah, ki jih imamo v programu.
Označimo z $e_i,\quad i=1,2,\ldots n$ standardne bazne vektorje v $\RR^n$ in z
$\II\subset\{1,2,\ldots n\}$ množico indeksov. Indeksi v $\II$ ustrezajo
posameznim spremenljivkam (indeksi določajo lokacijo spremenljivke v spominu). Naj bo $V_\II$ vektorski
podprostor napet na vektorje $\{e_i;\quad i\in\II\}$. Označimo z 
$\mathcal{P}_\II$ projekcijo na $V_i$, z $\mathcal{I}_\II$ pa vložitev $V_\II$ v
$\RR^n$. Vložitev $\mathcal{I}_\II$ ni mišljena kot linearna preslikava, ampak
dopuščamo, da so vrednosti spremenljivk, ki niso v $\II$, poljubne. Recimo, da nas zanima nabor spremenljivk z indeksi $\II$, ki jih
program $P$ izračuna na podlagi nabora spremenljivk z indeksi $\mathbb{J}$.
Preslikavo med $V_\JJ\to V_\II$, ki jo določa program $P$, lahko zapišemo kot
\begin{equation}
  \label{eq:zozitev}
  P^{\JJ}_{\II}=\mathcal{P}_\II\circ P\circ \mathcal{I}_\JJ 
\end{equation}
Odvod preslikave $ P^{\JJ}_{\II}$ je enak produktu
\begin{equation*}
  P^{\JJ}_{\II}=P_\II\cdot DP\cdot I_\JJ,   
\end{equation*}
kjer sta $P_\II$ in  $I_\JJ$ matriki, ki ustrezata projekciji na $V_\II$ oziroma
vložitvi $V_\JJ$ v $\RR^n$.



\subsection{Kontrolne strukture}

Do sedaj smo se omejili na operacije, ki spreminjajo vsebino spomina. Poleg
prireditvenih ukazov, poznamo tudi kontrolne ukaze (npr. stavki \texttt{if},
\texttt{for}, \texttt{while}, ...). Kontrolni stavki ne vplivajo neposredno na
vrednost spremenljivk, ampak spreminjajo potek programa. Seveda bo to vplivalo
tudi na odvod. Ampak za določen nabor vhodnih spremenljivk, bo potek programa
vedno enak. Zato si lahko kontrolne strukture predstavljamo kot definicijo
zlepka. Vzemimo naprimer preprost program v programskem jeziku Python
\begin{verbatim}
def abs(x):
  if x<0:
   return -x
  else
   return x
\end{verbatim}
Funkcija \texttt{abs(x)} je program, ki izračuna zlepek
\begin{equation}
  \label{eq:zlepek}
  |x| =
  \begin{cases}
    -x;\quad x<0\\
    x;\quad x\ge 0
  \end{cases}
\end{equation}
Vsaka kontrolna struktura razdeli prostor parametrov na različna območja,
znotraj katerih je potek programa enak. Celoten program torej razdeli prostor
vseh možnih parametrov na končno množico območij $\{\Omega_i;\quad i=1,\ldots
k\}$, kjer je potek programa enak. Program lahko torej v splošnem definiramo kot
zlepek. Za $\vec{x}\in\RR^n$ je
\begin{equation}
  \label{eq:zlrprk_splosno}
  P(\vec{x}) =
  \begin{cases}
    P_{n_11}\circ P_{(n_1-1)1}\circ\ldots P_{11}(\vec{x});&\quad \vec{x}\in\Omega_1\\
    P_{n_22}\circ P_{(n_2-1)2}\circ\ldots P_{12}(\vec{x});&\quad \vec{x}\in\Omega_2\\
    \vdots&\quad\vdots\\
    P_{n_kk}\circ P_{(n_k-1)k}\circ\ldots P_{1k}(\vec{x});&\quad \vec{x}\in\Omega_k\\
  \end{cases}
\end{equation}
Linearnizacija $TP$ programa $P$ je seveda tudi odvisna od začetnih parametrov
$\vec{x}$ in jo tudi lahko podamo kot zlepek a le v notranjosti deinicijskih območij $\Omega_i$
\begin{equation}
  \label{eq:zlrprk_splosno}
  TP_{\vec{x}} =
  \begin{cases}
    TP_{n_11}\cdot TP_{(n_1-1)1}\cdot\ldots TP_{11};&\quad \vec{x}\in\interior(\Omega_1)\\
    TP_{n_22}\cdot TP_{(n_2-1)2}\cdot\ldots TP_{12}(\vec{x});&\quad \vec{x}\in\interior(\Omega_2)\\
    \vdots&\quad\vdots\\
    TP_{n_kk}\cdot TP_{(n_k-1)k}\cdot\ldots TP_{1k}(\vec{x});&\quad \vec{x}\in\interior(\Omega_k)\\
  \end{cases}
\end{equation}
Problem je seveda na robu območju $\partial\Omega_i$, kjer program $P$ ni nujno odvedljiv.
\subsection{Direktno odvajanje}
\subsection{Obratno odvajanje}
\end{document}
