\documentclass{article}

\usepackage[english]{babel}
\usepackage{amsmath}
\usepackage{amsfonts}
\usepackage{amsthm}
\usepackage{caption}
\usepackage{bbm}

\usepackage[T1]{fontenc}
\usepackage{inconsolata}

\usepackage{tikz}
\usepackage{tikz-cd}
\usetikzlibrary{shapes,arrows}
\usetikzlibrary{positioning}

\usepackage{algorithm}
\usepackage{algpseudocode}
\usepackage{pifont}

\usepackage{color}
\definecolor{bluekeywords}{rgb}{0.13,0.13,1}
\definecolor{greencomments}{rgb}{0,0.5,0}
\definecolor{redstrings}{rgb}{0.9,0,0}

\usepackage{listings}

\lstset{language=[GNU]C++,
showspaces=false,
showtabs=false,
breaklines=true,
showstringspaces=false,
breakatwhitespace=true,
escapeinside={(*@}{@*)},
commentstyle=\color{greencomments},
keywordstyle=\color{bluekeywords}\bfseries,
stringstyle=\color{redstrings},
basicstyle=\ttfamily
}

\newcommand{\RR}{\mathbb{R}}
\newcommand{\Shift}{\mathcal{S}}
\newcommand{\II}{\mathbb{I}}
\newcommand{\JJ}{\mathbb{J}}
\newcommand{\E}{\mathcal{E}}
\newcommand{\T}{\mathcal{T}}
\newcommand{\VV}{\mathcal{V}}
\newcommand{\MM}{\mathcal{M}}
\newcommand{\NN}{\mathcal{N}}
\newcommand{\e}{\mathbf{e}}
\newcommand{\x}{\mathbf{x}}
\newcommand{\m}{\mathbf{m}}
\newcommand{\uu}{\mathbf{u}}
\newcommand{\vv}{\mathbf{v}}
\newcommand{\F}{\mathcal{F}}
\newcommand{\X}{\mathcal{X}}
\newcommand{\CC}{C\nolinebreak\hspace{-.05em}\raisebox{.4ex}{\tiny\bf +}\nolinebreak\hspace{-.10em}\raisebox{.4ex}{\tiny\bf +}}
\def\CC{{C\nolinebreak[4]\hspace{-.05em}\raisebox{.4ex}{\tiny\bf ++}}}
\newcommand{\dP}{\mathcal{P}}
% operator odvoda
\newcommand{\D}{\partial}
%operator 1 + \D
\newcommand{\DD}{\mathcal{D}}
% operator 1+ \D + \D^2 + ...
\newcommand{\sumd}{\tau}
\newcommand{\Op}{\partial^{\bigoplus}}
\newcommand{\op}[1]{\partial^{#1\bigoplus}}
\DeclareMathOperator{\interior}{int}


\newtheorem{definicija}{Definition}[section]
\newtheorem{trditev}{Claim}[section]
\newtheorem{izrek}{Theorem}[section]
\newtheorem{opomba}{Remark}[section]
\newtheorem{corollary}{Corollary}[section]

\usepackage[sorting=none,backend=bibtex]{biblatex}
\bibliography{biblio}


\title{Operational calculus and algebra of Algorithms}
\author{\v{Z}iga Sajovic, Martin Vuk}
\begin{document}
\maketitle
\begin{abstract}
In this paper, we develop the theory of analytic virtual machines, that
implement infinitely-differentiable programming spaces and operators acting upon them.

A programming space is a subalgebra of the function algebra of maps on the virtual
memory. We can construct a differential operator on programming spaces as we 
model virtual memory as a tensor product of a virtual space with tensor algebra
of its dual. Virtual memory serves by itself as an algebra of polynomial
programs, giving the approximation of the original program as a Taylor series at
program's input values. 

We present a theory of operators on programming spaces, that enables analysis of programs
and computations on the operator level, which favors general implementation. The theory enables
approximation and transformations of programs to arbitrary function basis', allowing us to choose programs' complexity, 
while knowing the order of the error of the approximation. We
also present several examples of how the theory can be used in computer science.

Theory opens new doors in program analysis. We develop a general
procedure which takes a program that tests an object for a property and
constructs a procedure that imposes that property upon any object. Through it we generalize state of the art methods for analyzing neural networks to general programs. Expanding upon them, we study computational dynamics as shape and the principles it induces upon the system.

\end{abstract}

\section{Introduction}
Programming holds the power of algorithmic control flow and freedom of expression, whose abstinence severely limits descriptiveness of closed form methods of \textit{pen and paper} mathematics, thus firmly cementing programming as the language of modern problem solving. Yet, a vibrant tradition of mathematics has existed since the dawn of mind, that remains, with the exception of combinatorics, largely untapped by computer science. 

Just as the discrete nature of physical reality is studied through analytic means, so can the nature of digital phenomena. Studying these procedures as objects undergoing change in some virtual space, has partially been attempted in some fields, such as Hamiltonian Monte Carlo methods of Bayesian predictors, that Girolami and Calderhead \cite{StatMC} studied as manifolds to great success, using unfortunately impractical methods of hard-coding derivatives of distributions. This of course stifles the freedom of algorithmic expression programming is embraced for.

The way evaluation of algorithmic expressions differs from evaluation of symbolic expressions of standard analysis, lies at the very core of this dissonance. The disconnect was partially mitigated by methods of automatic differentiation, utilized today in machine learning, engineering, simulations, etc. Yet under the lack of a proper formalism the model collapses when one tries to generalize to such notions as a differentiable program $p_1$ operating on (differentiable) derivatives of another program $p_2$ (where only coding of $p_2$ is required within $p_1$), while retaining the famed expressive freedom. 
Models allowing for nested differentiation \cite{AD1}, still fail in providing algorithms with an algebra enabling study and formulation of programs through analytic equations. Thus, they remain nothing more than means to calculating derivatives, void of any meaningful algebraic insight, lacking the true power the vast field of analysis is endowed with.

The aim of this paper is bridging the gap between programming and analysis. By generalizing on existing models, we show them to be specific views of the same great elephant. Employing tensor algebra, we construct a virtual memory whose internal structure reflects and carries evaluation of algorithmic expressions. In Section \ref{sec:operational} we provide an exact definition and construction of an analytic virtual machine, capable of implementing infinitely-differentiable programming spaces and operators acting upon them, supporting actions on multiple memory locations at a time. These programming spaces are shown to be a subalgebra, giving rise to algebraic manipulations of programs and through it operators expanding a program into an infinite tensor series are derived. The operator of program composition is constructed, generalizing both forward and reverse mode of automatic differentiation (of arbitrary order) under a single invariant operator in the theory. Through projections and embeddings in this space, the problem of nested differentiation is resolved, where a program $p_1$ acts on differentiable derivatives of $p_2$, requiring only coding of $p_2$ within $p_1$.

Theory grants us the ability to choose programs' complexity through approximations inhabiting the virtual space. It being modeled by tensor algebra consisting of multi-linear maps is tailor made for efficient implementation by GPU parallelism \cite{TensorGPU}. Employing the developed algebra, we derive functional transformations of programs in an arbitrary function basis. As different hardware is optimized for running of different sets of functions, this proves useful with adapting code to fit specific hardware.
We provide instructions on methods of usage and outline the process of how these transformations are to be interchangeably applied in practice.

The paper offers new insights into programs, as we demonstrate how to probe their inner structure, revealing what actions properties of the domain most react to. This enables us to extract archetypes from data, or alter it by imposing some property it lacks upon it.
We generalize state of the art methods for analyzing neural networks \cite{DeepDream} to general programs and provide a framework for analysis of machine learning architectures and other virtual constructs, as actions on the virtual space. Exploring procedures as systems inducing change upon objects in this space, we study computational dynamics as shape and the principles it induces upon the system and attain valuable tools for general problem solving.

\section{Computer programs as maps}
We will model computer programs as maps on a vector space. If
we only focus on the real valued variables (of type \texttt{float} or
\texttt{double}),  the state of the virtual memory can be seen as a high
dimensional vector\footnote{we assume the variables of interest to be of type \texttt{float} for
  simplicity. Theoretically any field can be used instead of $\RR$.}. 
A set of all the possible states of the program's memory,
can be modeled by a finite dimensional real vector space $\VV\equiv \RR^n$. We
will call $\VV$ the \emph{memory space of the program}. The effect of a computer
program on its memory space $\VV$, can be described by a map
\begin{equation}
  \label{eq:map}
  P:\VV\to \VV.
\end{equation}
We can compose two programs sequentially, to obtain another program, which
results in map composition. The set of all such maps $\F=\VV^\VV$ is a
\emph{monoid} for the operation of composition $\circ$. A programing space is
a space spanned by a set of elementary operations that are sequentially applied into programs. 
\begin{definicija}[Euclidian virtual machine] The tuple $\langle \VV,\F\rangle$ is an Euclidian virtual machine.
  \begin{itemize}
  \item
  $\VV$ is a finite dimensional vector space over a filed $K$ serving as alphabet symbols
  \item
  $\F= \VV^\VV$ is the space of maps $P:\VV\to \VV$, called \emph{programming space}, serving as states
  \end{itemize}  
\end{definicija}

Denote by $e_1,\ldots e_n$ a standard basis of the memory space $\VV$ and by
$x_1,\ldots x_n$ the dual basis of $\VV^*$. The functions $x_i$ are coordinate
functions on $\VV$ and correspond to individual variables in the memory. For
each program $P:\VV\to \VV$ we divide variables $x_i$ into three groups. Let us
denote with $P_i$ the $i-th$ component of the map $P$ according to the standard
basis. Function $P_i:\VV\to K$ is a scalar function and is equal to $x_i\circ P$.
\begin{definicija}
  Let $x_i$ be the variable in memory and $P:\VV\to\VV$ a program.  
  \begin{enumerate}
  \item 
    The variable $x_i$ is called \emph{free}, if
    \begin{equation}
      \label{eq:free}
      \frac{\partial P_j}{\partial x_i}\equiv \delta_{ij}\wedge \frac{\partial P_i}{\partial x_j}\equiv \delta_{ij}.
    \end{equation}
\item 
    The variable $x_i$ is called \emph{input parameter} if
    \begin{equation}
      \label{eq:input}
      \frac{\partial P_j}{\partial x_i}\not\equiv \delta_{ij}.\footnote{this statement also includes the case that the partial derivative do not exist at some point}
    \end{equation}
\item  The variable $x_i$ is called \emph{output value} if
    \begin{equation}
      \label{eq:output}
      \frac{\partial P_i}{\partial x_j}\not\equiv \delta_{ij} 
    \end{equation}
  \end{enumerate}

\end{definicija}
Free variables are those that are left intact by the program and have no
influence on the final result. The output of the program depends only on the
values of the input variables and consists of variables that are changed during
the program. Input parameters and output values might overlap. 
We also define corresponding vector subspaces of the program $P$.
\begin{definicija}
  The \emph{input} or \emph{parameter space} is a vector subspace $I_P<\VV$,
  spanned by the standard basis vectors, that correspond to \emph{input}
  variables. Similarly  \emph{output space} is a subspace
  $O_P<\VV$ is spanned by basis vectors corresponding to output variables and
  \emph{free space} is a subspace $F_P<\VV$ spanned by the basis vectors
  corresponding to free variables. 
\end{definicija}
The input and the output space of a program $P$ are the minimal
vector subspaces spanned by the standard basis vectors, such that the map $P_e$, 
defined by the following commutative diagram 
\begin{equation} 
    \label{eq:induced_map}
\begin{tikzcd}
  \VV \arrow{r}{P} & 
  \VV \arrow{d}{\mathrm{pr}_{O_P}}\\
  I_P \arrow[hook]{u}{\vec{i}\mapsto \vec{i}+\vec{f}} 
  \arrow{r}{P_e}& O_P
\end{tikzcd}
  \end{equation}
does not depend of the choice of the element 
$\vec{f}\in F_P=(I_P+O_P)^\perp$. The map $P_e$ describes the actual effect of the
program $P$, since it ignores the free memory. 
\section{Multivariate derivatives}
\begin{definicija}[Derivative]
  Let $V,U$ be Banach spaces. The map $P:V\to U$ is differentiable at the point
  $x\in V$, if there exists a linear bounded operator $TP_x:V\to U$ such that
  \begin{equation}
    \label{eq:frechet}
    \lim_{h\to 0}\frac{\|P(x+h)-P(x)-TP_x(h)\|}{\|h\|} = 0.
  \end{equation}
  The map $TP_x$ is called the \emph{Fréchet derivative} of the map $P$ at the
  point $\x$.
\end{definicija}
For maps $\RR^n\to \RR^m$ Fréchet derivative can be expressed by multiplication
of vector $h$ by the Jacobi matrix $\mathrm{J} P$ of the  partial derivatives of the 
components of the map $P$
\begin{equation*}
  TP_x(h) = \mathrm{J} P(x)\cdot h.
\end{equation*}
We assume for the remainder of this section that the map $P:V\to U$ is
differentiable for all $\x\in V$. The derivative defines a map from $V$ to
linear bounded maps from $V$ to $U$. We further assume $U$ and $V$ are finite
dimensional. Then the space of linear maps from $V$ to $U$ is isomorphic to
tensor product $U\otimes V^*$. For a simple tensor $\uu\otimes f\in U\otimes
V^*$ one can define a linear map by tensor contraction 
 \begin{equation}
   \label{eq:lin_tenzor}
   \uu\otimes f:\x \mapsto f(\x)\cdot \uu.
 \end{equation}
The above formula can be extended by linearity to any tensor, making the isomorphism
between tensor product $U\otimes V^*$ and the space of linear maps $V\to U$. The derivative defines a map
\begin{eqnarray}
  \label{eq:odvod_preslikava}
  \D P&:& V\to U\otimes V^*\\
  \D P&:& \x \mapsto T_\x P.
\end{eqnarray}
One can consider the differentiability of the map \eqref{eq:odvod_preslikava}.
The derivative of the map $\D P$ is called the \emph{second derivative}
of the map $P$ and is denoted by $\D^2P$. Second derivative is at each point a
linear map  $V\to U\otimes V^*$, and can be therefore 
represented by a tensor from $(U\otimes V^*)\otimes V^*$. Second derivative
defines a map 
$$\D^2 P:V\to U\otimes V^*\otimes V^*.$$ 

Similarly higher derivatives can be defined as maps.
\begin{definicija}[higher derivatives]
  Let $P:V\to U$ be a map from vector space $V$ to vector space $U$. 
The derivative of order $k$ is a map $\D^kP:V\to U\otimes(V^*)^{\otimes k}$,
that maps $\x\in V$ to the derivative  $T_\x(\D^{k-1}P)$ of the $k-1$-th order
derivative as a map $\D^{k-1}P: V\to U\otimes (V^*)^{\otimes k-1}$:
  \begin{eqnarray}\label{eq:partial}
    \label{eq:visji_odvod}
    \D^kP&:&V\to U\otimes (V^*)^{\otimes k}\\
    \D^kP&:&\x\mapsto T_\x\left( \D^{k-1}P \right)
  \end{eqnarray}
\end{definicija} 
\begin{opomba}
  For the sake of clarity, we assumed in the definition above, that the map $P$ as well as all its
  derivatives are differentiable at all points $\x$. If this is not the case
  definitions above can be done locally, which would introduce mostly technical difficulties.
\end{opomba}
Let $\e_1,\ldots,\e_n$ be a basis of $U$ and $x_1,\ldots x_m$ the basis of
$V^*$. Then $\D^kP$  can be defined in terms of
directional(partial) derivatives by the formula
\begin{equation}\label{eq:d}
	\partial^kP=\sum_{\forall_{i,\alpha}}\frac{\partial^k P_i}{\partial
	    x_{\alpha_1}\ldots \partial x_{\alpha_k}}\e_i\otimes
	  dx_{\alpha_1}\otimes\ldots \otimes dx_{\alpha_k}.
\end{equation}

\subsection{Differentiable programs}
To each program $P:\VV\to \VV$ one can attach its effective map $P_e:I_P\to
O_P$. The derivative of this map is of interest, when we speak about
differentiability of computer programs. 
\begin{definicija}[Automatically differentiable programs]
  A program $P:\VV\to \VV$ is \emph{automatically differentiable} if there exist an embedding
  $O_P\otimes I_P^*\hookrightarrow F_P$ of the space $O_P\otimes I_P^*$ into
  the free space $F_P$, and a program $\D P:\VV\to \VV$, such that its effective
  map is the map
  \begin{equation}
    \label{eq:program_derivative}
    P_e\oplus \D P_e:I_P\rightarrow O_P\oplus (O_P\otimes I^*).
  \end{equation}
\end{definicija}

If a program $P:\VV\to \VV$ is automatically differentiable then it is also
differentiable as a map $\VV\to\VV$. However only the derivative of program's
effective map can be implemented as a program, since the memory space is limited to $\VV$. 

Putting memory considerations aside, we propose an universal model of the memory
space to be 
$$\VV_\infty = \VV\otimes T(\VV^*) = \VV\oplus
(\VV\otimes\VV^*)+\ldots.$$
We will call $\VV_\infty$ the \emph{universal virtual memory space} of a virtual computing machine. The term virtual memory is used as it is
only possible to embed certain subspaces of $\VV_\infty$ into memory space $\VV$, making it similar to
virtual memory as a memory management technique.  

We extend each program $P:\VV\to \VV$ to the map on universal memory space $\VV_\infty$ by setting the first component to $P$,
and all other components to zero. Each element of $\VV_\infty$ can be written
uniquely as a sum $\vv_0+T_1+T_2+\ldots$, where $\vv_0\in\VV$ and 
$T_i\in\VV\otimes (\VV^*)^{\otimes i}$ and the extended map acts on it as follows   
\begin{equation}
  \label{eq:extension}
  P(\vv_0+T_1+T_2+\ldots)=P(\vv_0)+0+0+\ldots.
\end{equation}

\section{Tensor algebra representation}
In order to apply the definition of derivative of a program to higher
derivatives, we make use of the tensor algebra.
Let $P:V\to U$ be a map on vector spaces. Motivated by the definition of higher
derivatives, we define a sequence of vector spaces with
recursive formula
\begin{eqnarray}
  \label{eq:universal_space}
  U_0 &=& U\\
  U_k &=& U_{k-1}+\left(U_{k-1}\otimes V^*\right).
\end{eqnarray}
Note that the sum is not direct, since some of the subspaces of $U_{k-1}$ and
$U_{k-1}\otimes V^*$ are naturally isomorphic and will be identified.

The formula can be illustrated as follows. 
To implement the derivative of a map $P:V\to U$, the space $U\otimes V^*$ has to
be allocated along with original co-domain $U$.  The space $U\oplus
(U\otimes V^*)$ has to be embedded into the memory space $\VV$.
Denote $U_k$ the space needed for $k$-th order derivative. The program for
$k-th$ derivative would implement the map $V\to U_k$. For the derivative of this map
the space $U_k+ (U_k\otimes V^*)$ would have to be allocated. This sum
however would not be direct, since some of the subspaces of $U_k$ coincide with
subspaces of $U_k\otimes V^*$.
\begin{trditev}
  The space
  \begin{equation}
    \label{eq:k-th-virtual-space}
    U_k = U\otimes \left(K\oplus V^* \oplus (V^*\otimes V^*)\oplus\ldots
      (V^*)^{\otimes k}\right) = U\otimes T_k(V^*)
  \end{equation}
  satisfies the recursive formula (\ref{eq:universal_space}), where the space
  $T_k(V^*)$ is a subspace of \emph{tensor algebra} $T(V^*)$, consisting of linear
  combinations of tensors of rank less or equal $k$.
\end{trditev}

\begin{proof}
  We prove the claim by induction. The base of induction is trivial, for the
  induction step we first find that:
  \begin{multline*}
    U\otimes T_k(V^*)=U\otimes(K\oplus V^*\oplus\ldots (V^*)^{\otimes (k+1)})=\\
    U\oplus (U\otimes V^*)\oplus\ldots (U\oplus U\otimes(V^*)^{\otimes (k+1)})
  \end{multline*}
  on the other hand, using recursive formula
  \begin{multline}
    U_k+(U_k\otimes V^*) = \left( U\oplus (U\otimes V^*)\oplus\ldots (U\otimes(V^*)^{\otimes k})\right) + \\
    \left((U\otimes V^*) \oplus (U\otimes V^*\otimes V^*) \oplus\ldots
      (U\otimes(V^*)^{\otimes k}\otimes V^*)\right)
  \end{multline}
  However spaces $U\otimes(V^*)^{\otimes (j+1)}$ and $U\otimes (V^*)^{\otimes
    j}\otimes V^*$ are naturally isomorphic and will be identified, leaving
  \begin{equation}
    U_k+(U_k\otimes V^*) = U\oplus (U\otimes V^*)\oplus\ldots (U\otimes(V^*)^{\otimes (k+1)})
  \end{equation}
\end{proof}
The spaces $U_k$ form an increasing sequence of subspaces (a flag) of the space $U\otimes
T(V^*)$
\begin{equation}
  \label{eq:flag}
  U_0<U_1<U_2<\ldots < U\otimes T(V^*).
\end{equation}
This enables us to define all the derivatives as maps with
the same domain and codomain $V\to U\otimes T(V^*)$. Indeed for any map $P:V\to
U$, one can define a map to $U\otimes T(V^*)$ as
$$V\xrightarrow{P}U\hookrightarrow U\otimes T(V^*)$$ and similarly  for all of its
derivatives
$$V\xrightarrow{\D^k P} U\otimes (V^*)^{\otimes k}\hookrightarrow U\otimes
T(V^*).$$

\section{Operational calculus and algebra of Algorithms}\label{sec:operational}

The derivative defines an operator on the space of smooth maps $V\to U\otimes
 T(V^*)$\footnote{the operator $\D$ may be defined partially for other maps as
   well.}. We denote this operator $\D$. If we look at a map $P:V\to U$ as a map
 to $U\otimes T(V^*)$, then the image of $P$ by operator $\D$ is the first
 derivative, while the higher order derivatives are just powers of operator $\D$
 applied to $P$. 

Let us define the following function spaces.
 \begin{equation}\label{eq:F^n}
 	\F^n=\{f:V\to U\otimes(V^*)^{n\otimes}\}
 \end{equation}
 and
 \begin{equation}\label{eq:F_n}
 	\F_n=\{f:V\to U\otimes T_n(V^*)\}
 \end{equation}
All of these function spaces are subspaces of $\F_\infty=\{f:V\to U\otimes
T(V^*)\}$. The space $\F_n$ is direct a sum of $\F^k$ for $k\le n$ and $\F_n<\F_m$ if $n<m$. We
have an increasing sequence of function spaces
\begin{equation}
  \label{eq:flag_function_space}
  \F_0<\F_1<\ldots<\F_n<\ldots
\end{equation}

We define the differential operator as
\begin{equation}\label{eq:dd}
	\partial^k=\sum_{\forall_{i,\alpha}}\frac{\partial^k}{\partial
	    x_{\alpha_1}\ldots \partial x_{\alpha_k}}\e_i\otimes
	  dx_{\alpha_1}\otimes\ldots \otimes dx_{\alpha_k}.
\end{equation}
Thus $\D^k$ is a mapping between function spaces $\eqref{eq:F^n}$
 \begin{equation}\label{eq:toFn+k}
 \D^k:\F^n\to\F^{n+k}
 \end{equation}
 
 \begin{trditev}\label{izr:linearnaNeodvisnost}
  The set $\DD^n=\{\partial^k;\quad 0\le k\le n\}$ represents linearly independent vectors over a field $K$.
 \end{trditev}
 
 \begin{definicija}\label{def:dP}
 	A \emph{differentiable programming space} $\dP_0$ is any subspace of $\F_0$ such that
 	\begin{equation}\label{eq:P}
 	\D\dP_0\subset\dP_0\otimes T(V^*)
 	\end{equation}
 	When all elements of $\dP_0$ are analytic, we denote $\dP_0$ as an \emph{analytic programming space}.
 \end{definicija}

\begin{izrek}\label{izr:P}
	Any differentiable programming space $\dP_0$ is an
  infinitely differentiable programming space, such that
	\begin{equation}\label{eq:P_n}
	 		\D^k\dP_0\subset\dP_0\otimes T(V^*)
	 	\end{equation}
for any $k\in\mathbb{N}$.
\end{izrek}
\begin{proof} By definition \ref{def:dP}
$$\D\dP_0\subset\dP_0\otimes T(V^*)$$
	Induction step is trivial. For all 	$\forall_{P\in\dP_0}$,
  $\D^n\dP_0\subset\dP_0\otimes T(V^*)$, where
  $\alpha$ is multiindex denoted the component of $T(V^*)$ and $i$ is the
  index denoting the component of $U$.
	\begin{equation}\label{eq:inductionStep}
\D^{n+1}P_{\alpha,k}^i=\D(\D^n P^i_\alpha)_k\land(\D^n P^i_\alpha)\in\dP_0\implies \D(\D^n P^i_\alpha)_k\in \dP_0\otimes T(V^*)
	\end{equation}
	$$\implies$$
	$$\D^{n+1}\dP_0\subset\dP_0\otimes T(V^*)$$
\end{proof}

 \begin{corollary}\label{izr:dP}
	A programming space $\dP=\{\dP:V\to U\otimes T(V^*)\}$, is closed under the differential operator $\D^k$.
 \end{corollary}
 \begin{proof}
 	 By $\eqref{eq:P_n}$ and the symmetric nature of the operator $\D$, for each $k$ there exists an isomorphism
 	 	\begin{equation}
 	 		\dP_0\to\D^k\dP_0
 	 	\end{equation}
 \end{proof}

By Theorem $\ref{izr:dP}$ we may represent calculation of derivatives of the map $P:V\to U$, with only one mapping. We define the geometric series
 
 \begin{equation}\label{eq:DD}
  	\sumd_n = 1+\D +\D^2 +\ldots + \D^n 
  \end{equation}
  
  
  \begin{equation}
  	\sumd_n=\sum\limits_{n=0}^{n}\sum_{\forall_{i,\alpha}}\frac{\partial^n}{\partial
  		    x_{\alpha_1}\ldots \partial x_{\alpha_k}}\e_i\otimes
  		  dx_{\alpha_1}\otimes\ldots \otimes dx_{\alpha_k} , x_i\in V_{j\in\JJ}
  \end{equation}
  
  By Theorem $\ref{izr:linearnaNeodvisnost}$ the operator $\sumd_n$ is unique and prescribes the map $P: V\to U$ with a map $\sumd_k:V\to
U\otimes T(V^*)$, composed of the map itself and all derivatives of order $n\le k$. The image $\sumd_kP(\x)$ is a multitensor of order $k$, which is a direct sum of the maps value and all derivatives of order $n\le k$, all evaluated at the point $\x$:
\begin{equation}
  \label{eq:multi_odvod}
  \sumd_kP(\x) = P(\x)+\D_\x P + \D^2_\x P + \ldots + \D^k_\x P.
\end{equation}
We seek to generalize such notions.

\begin{definicija}\label{def:P_n}
Let $\dP_n$ be the subspace, spanned by $\DD^n\dP_0$ over $K$.
 \end{definicija}
 
 \begin{izrek}\label{izr:P_n}
 	Function space $\dP_n:V\to U\otimes T(V^*)$ can be embedded into the tensor
  product of the function space $\dP_0:V\to U$ and the subspace $T_n(V^*)$
  tensor algebra of the dual od the virtual space $V$. 
 \end{izrek}
 
 \begin{proof} Follows directly from the proof of Theorems $\ref{izr:P}$ and $\ref{izr:dP}$ through argument $\eqref{eq:inductionStep}$.
  \end{proof}
By taking the limit as $n\to \infty$, we see
 	
 	\begin{equation}
 	\label{eq:P_algebra}
 	 	    \dP_\infty < \dP_0\otimes T(V^*) = \dP_0 \otimes\left(\bigoplus_{k=0}^\infty (V^*)^{\otimes k} \right).
 	\end{equation}

  \begin{corollary}
  If there exists a construction of a first order programming space $\dP_1$ by a
  single map $\sumd_1$, such that $\dP_1=\sumd_1\dP_0$, there exists a
  construction of a programming space of an arbitrary order $\dP_n$ by a map
  $\sumd_n$, defined by the recursion formula.
  \begin{equation}
    \label{eq:potenca(1+d)}
    \sumd_{k+1}=1+\D\sumd_{k}
  \end{equation}

  \end{corollary}
  \begin{proof}
    The recursive relation between $\sumd_k$ and $\sumd_{k+1}$ is
    \begin{equation}
      \label{eq:rekurzija}
      \sumd_{k+1} = 1 + \D +\D^2+\ldots \D^{k+1} = 1+\D(1+\D+\ldots +\D^{k}) = 1+\D\sumd_k.
    \end{equation} 

  \end{proof}
       
    The following
     \begin{equation}
     	\sumd_n\dP_n\subset\dP_0\otimes T_n(V^*)
     \end{equation}
   	holds by Theorem $\ref{izr:P_n}$, allowing simple implementation, as dictated by expression $\eqref{eq:potenca(1+d)}$. Following $\eqref{eq:potenca(1+d)}$ and the argument $\eqref{eq:inductionStep}$ of the proof of Theorem \ref{izr:P}, it is evident that only explicit knowledge of the operator $\sumd_1$ is required.
        Of course the same holds true for all generating maps of the type
        $$\dP_0\to\dP_n.$$
 \begin{corollary}
   If all elements of $\dP_0$ are analytic, than so are the elements of $\dP_n$.
   \end{corollary}        
        
       \begin{opomba}\label{rem:vTen}
       Maps $V\otimes T(V^*)\to V\otimes T(V^*)$ are constructible through tensor algebra on $\dP_n$ and through it, compositions of programs in $\dP_n$.
       \end{opomba}
       
 

\begin{definicija}[Algebra product]
 For any bilinear map $\cdot :V\times V\to U\otimes T(V^*)$ we can define a
 bilinear product $\cdot$ on $U\otimes T(V^*)$ by the following rule on the simple
 tensors:
 \begin{eqnarray}
   \label{eq:algebra_product}
   (\vv\otimes f_1\otimes\ldots f_k) \cdot (\uu\otimes g_1\otimes\ldots) &=& 
f_k(\uu)\vv\otimes f_1\otimes\ldots f_{k-1}\otimes g_1\otimes\ldots \\
   \vv\cdot (\uu\otimes g_1\otimes\ldots g_l) &=&  (\vv\cdot \uu)\otimes
    \nonumber                                              g_1\otimes\ldots g_l
 \end{eqnarray}
\end{definicija}
\begin{izrek}[Algebra of Algorithms]\label{izr:alg}
An infinitely-differentiable programming space $\dP_n$ is a function algebra,
with the product defined by \eqref{eq:algebra_product} for any bilinear map $\cdot :V\times V\to U\otimes T(V^*)$.
\end{izrek}

\subsection{Analytic virtual machine}
Algorithms may be interpreted as a function algebra over $\dP_0:\VV\to \VV$.
By Theorem $\ref{izr:P_n}$, we can express $\dP_n$, that is spanned by $\sumd_n\dP_0$ over $K$, through linear combinations of elements of $\dP_0\otimes T(V^*)$. To construct $\dP_n$, only the tuple $(\dP_0,\VV)$, equipped with tensor algebra is required. Thus the tuple  $(\dP_0,\VV)$ and the belonging algebra are sufficient conditions for the existence and construction of infinitely differentiable programing spaces $\dP_n$ $\eqref{def:P_n}$, which are function algebras by Theorem $\ref{izr:alg}$.

\begin{definicija}[Analytic virtual machine]
The tuple $M=\langle \VV,\dP_0\rangle$ is an analytic, infinitely  differentiable virtual machine.
   
    \begin{itemize}
    \item
    $\VV$ is a finite dimensional vector space
    \item
    $\VV\otimes T(\VV^*)$ is the virtual memory space, serving as alphabet symbols
    \item
    $\dP_0$ is an analytic programming space over $\VV$
    \end{itemize}
    When $\dP_0$ is a differentiable programming space, this defines an infinitely-differentiable virtual machine.
  \end{definicija}
Anything that operates under these specifications is an analytic virtual machine, with the ability to implement infinitely differentiable programming spaces $\dP_\infty$. 

 \subsection{Power series expansion}\label{sec:Vrsta}
 
We motivate the following with an example. Suppose an algorithm of high time complexity. As speed is of importance, an approximation would suffice. Using the described mechanics, we may linearize the program. We choose a $v_0\in\dP$ and linearize $P'=P(v_0)+\D P(v_0)\cdot(v-v_0)$. The latter has to be done only once.
 Locally, the image of the linearization is a good approximation of the original program. To extend this radius, we would like to expand said program into a series. Furthermore, expansion into a series offers new insights into programs, that are studied in the following sections. 
 
 By Claim $\ref{izr:linearnaNeodvisnost}$  there exists a space that is spanned by the set $\{\D^k\}$ over a field $K$. Thus, the expression
 \begin{equation}
 	e^{h\D}=\sum\limits_{n=0}^{\infty}\frac{(h\D)^n}{n!}
 \end{equation}
 is well defined. In coordinates, the operator $e^{h\D}$ can be written as a
 series over all multi-indices $\alpha$
 \begin{equation}\label{eq:e^d}
 	e^{h\D}=\sum\limits_{n=0}^{\infty}\frac{h^n}{n!}\sum_{\forall_{i,\alpha}}\frac{\partial^n}{\partial
 		    x_{\alpha_1}\ldots \partial x_{\alpha_n}}\e_i\otimes
 		  dx_{\alpha_1}\otimes\ldots \otimes dx_{\alpha_n}.
 \end{equation}
The operator $e^{h\D}$ is a mapping between function spaces $\eqref{eq:F_n}$
 \begin{equation}
 	e^{h\D}:\dP\to\dP_\infty.
 \end{equation}
 It also defines a map
  \begin{equation}\label{eq:specProg}
  	e^{h\D}:\dP\times V\to U\otimes T(V^*),
  \end{equation}
by taking the image of the map $e^{h\D}(P)$ at a certain point $\vv\in V$.  
Through $\eqref{eq:specProg}$ we may construct a map from the space of programs,
to the space of polynomials. Note that the space of multivariate polynomials
$V\to K$ is isomorphic to symmetric algebra $S(V^*)$, which is in turn a
quotient of tensor algebra $T(V^*)$. To any element of
 $U\otimes T(V^*)$ one can attach corresponding element of $U\otimes S(V^{i*})$
 namely a polynomial map  $V\to U$.
 \begin{equation}\label{eq:pToPol}
 	e^{h\D}: \dP\times V\to U\otimes S(V^*)
 \end{equation}
 For any element $\vv_0\in V$, the expression $e^{h\D}(\cdot,\vv_0)$ is a map $\dP\to
 U\otimes S(V^*)$, mapping a program to a polynomial. We can express the
 correspondence between multi-tensors in $U\otimes T(V^*)$ and polynomial maps
 $V\to U$ given by multiple contractions for all possible indices. For a simple tensor $\uu\otimes
 f_1\otimes\ldots\otimes f_n\in U\otimes(V^*)^{\otimes n}$ the contraction by
 $\vv\in V$ is given by applying co-vector $f_n$ to $\vv$ 
 \begin{equation}
   \label{eq:contraction}
 \uu\otimes f_1\otimes\ldots\otimes f_n\cdot \vv = f_n(\vv) \uu\otimes f_1\otimes\ldots f_{n-1}.
 \end{equation}
Applying contraction multiple times with the same vector $\vv$ one can assign to
any simple tensor a monomial map 
 \begin{equation}
   \label{eq:tensor->poly}
 \uu\otimes f_1\otimes\ldots\otimes f_n: \vv \mapsto f_n(\vv)\cdots f_1(\vv) \uu
 \end{equation}

and by linearity to any finite rank multi-tensor in $U\otimes T(V^*)$ a
polynomial map.
 
\begin{izrek}\label{izr:e^d}
	For a program $P\in\dP$  the expansion into a Taylor series
  at the point $\vv_0\in V$ is expressed by multiple contractions 
	\begin{multline}\label{eq:tenzorVrsta}
	P(\vv_0+h\vv) = \Big((e^{h\D}P)(\vv_0)\Big)\cdot\vv
  = \sum_{n=0}^\infty\frac{h^n}{n!}\D^nP(\vv_0)\cdot (\vv^{\otimes n})\\
  = \sum_{n=0}^\infty \frac{h^n}{n!}\sum_{\forall_{i,\alpha}}\frac{\partial^nP_i}{\partial
 		    x_{\alpha_1}\ldots \partial x_{\alpha_n}}\e_i\cdot
 		  dx_{\alpha_1}(\vv)\cdot\ldots \cdot dx_{\alpha_n}(\vv).
	\end{multline}
\end{izrek}
 
 \begin{proof}
We will show that $\frac{d^n}{dh^n}\text{(LHS)}|_{h=0}=\frac{d^n}{dh^n}\text{(RHS)}|_{h=0}$. Then $\text{LHS}$ and $\text{RHS}$ as functions
of $h$ have coinciding Taylor series and are therefore equal.\\
 $\implies$
 
 $$\left. \frac{d^n}{dh^n}P(\vv_0+h\vv)\right|_{h=0}=\D^n P(\vv_0)(\vv)$$
 $\impliedby$
 $$\left. \frac{d^n}{dh^n}\left((e^{h\D})(P)(\vv_0)\right)(\vv)\right|_{h=0}=
\left. \left((\D^n e^{h\D})(P)(\vv_0)\right)(\vv)\right|_{h=0}$$
 $$\land$$
 $$\left. \D^ne^{h\D}\right| _{h=0}=\left. \sum\limits_{i=0}^{\infty}\frac{h^i\D^{i+n}}{i!}\right|_{h=0}=\D^n$$
 $$\implies$$
 $$\left(\D^n(P)(\vv_0)\right)(\vv^{\otimes n})$$
 \end{proof}
 
 The image of the contraction is an element of the original virtual space $V^i$. Independence of the operator $(\ref{eq:specProg})$ from a coordinate system, translates to independence in execution. Thus the expression $(\ref{eq:tenzorVrsta})$ is invariant to the point in execution of a program, a fact we explore later on.  
 
 Theorem $\ref{izr:P_n}$ implies
     \begin{equation}
     	e^{h\D}(\dP_0)\subset\dP_0\otimes T(V^*)
     \end{equation}      
which enables efficient implementation.
 
 \begin{izrek}\label{izr:prod}
 The operator $e^{h\D}$ is an automorphism of the algebra over $\dP_\infty$
 \begin{equation}
 	e^{h\D}(p_1\cdot p_2)=e^{h\D}(p_1)\cdot e^{h\D}(p_2)
 \end{equation}
 where $\cdot$ stands for a bilinear map.
 \end{izrek}
 
 \begin{proof}
 We will show that $\frac{d^n}{dh^n}\text{(LHS)}|_{h=0}=\frac{d^n}{dh^n}\text{(RHS)}|_{h=0}$. Then $\text{LHS}$ and $\text{RHS}$ as functions
 of $h$ have coinciding Taylor series and are therefore equal.\\
  $\implies$
  $$\lim\limits_{\lVert h\rVert\to 0}(\frac{d}{dh})^ne^{h\D}(p_1\cdot p_2)=\lim\limits_{\lVert h\rVert\to 0}\D^ne^{h\D}(p_1\cdot p_2)$$
  $$\implies$$
  \begin{equation}\label{eq:Leibnitz1}
  \D^n(p_1\cdot p_2)
  \end{equation}
  $\impliedby$
  $$\lim\limits_{\lVert h\rVert\to 0}(\frac{d}{dh})^n\Big(e^{h\D}(p_1)\cdot e^{h\D}(p_2)\Big)=\lim\limits_{\lVert h\rVert\to 0}\sum\limits_{k=0}^{n}{n\choose k}\D^{n-k}e^{h\D}(p_1)\cdot \D^ke^{h\D}(p_2)$$
  $$\implies$$
  \begin{equation}\label{eq:Leibnitz2}
  \sum\limits_{k=0}^{n}{n\choose k}\D^{n-k}p_1\cdot \D^kp_2
  \end{equation}
  by Leibnitz rule for higher derivatives of bilinear maps \eqref{eq:Leibnitz1} equals \eqref{eq:Leibnitz2}.
 \end{proof}
 
 \begin{izrek}\label{izr:kompo}
 Composition of maps $\dP$ is expressed as
 \begin{equation}\label{eq:kompo}
 e^{h\D}(f\circ g)=\exp(\D_fe^{h\D_g})(g,f)
 \end{equation}
 where $\exp(\D_fe^{h\D_g}):\dP\times\dP\to\dP_\infty$ is an operator on pairs of maps $(g,f)$, where $\D_g$ is applied to $g$, and $\D_f$ to $f$. 
 \end{izrek}
 
\begin{proof}
  We will show that $\frac{d^n}{dh^n}\text{(LHS)}|_{h=0}=\frac{d^n}{dh^n}\text{(RHS)}|_{h=0}$. Then $\text{LHS}$ and $\text{RHS}$ as functions
  of $h$ have coinciding Taylor series and are therefore equal.\\
 $\implies$
 $$\lim\limits_{\lVert h\rVert\to 0}(\frac{d}{dh})^ne^\D(f\circ g)=\lim\limits_{\lVert h\rVert\to 0}\D^ne^{h\D}(f\circ g)$$
 $$\implies$$
 \begin{equation}\label{eq:kompproof1}
 \D^n(f\circ g)
 \end{equation}
 $\impliedby$
 $$\exp(\D_fe^{h\D_g})=\exp\left(\D_f\sum\limits_{i=0}^{\infty}\frac{(h\D_g)^i}{i!}\right)=\prod_{i=1}^{\infty}e^{\D_f\frac{(h\D_g)^i}{i!}}\Big(e^{\D_f}\Big)$$
 $$\implies$$
 $$\exp(\D_fe^{h\D_g})(g)(f)=\sum\limits_{\forall_n}h^n\sum\limits_{\lambda(n)}\prod\limits_{k\cdot l\in\lambda}\Big(\frac{\D_f\D_g^l(g)}{l!}\Big)^k\frac{1}{k!}\Big(\Big(e^{\D_f}\Big)f\Big)$$
 where $\lambda(n)$ stands for the partitions of $n$.
 $$\lim\limits_{\lVert h\rVert\to 0}(\frac{d}{dh})^n\exp(\D_fe^{h\D_g})=\lim\limits_{\lVert h\rVert\to 0}\sum\limits_{\forall_m}n!h^{n-m}\sum\limits_{\lambda(m)}\prod\limits_{k\cdot l\in\lambda}\Big(\frac{\D_f\D_g^l(g)}{l!}\Big)^k\frac{1}{k!}\Big(\Big(e^{\D_f}\Big)f\Big)$$
 $$\implies$$
 \begin{equation}\label{eq:dComposite}
 \sum\limits_{\lambda(n)}\prod\limits_{k\cdot l\in\lambda}\Big(\frac{\D_f\D_g^l(g)}{l!}\Big)^k\frac{n!}{k!}\Big(\Big(e^{\D_f}\Big)f\Big)
 \end{equation}
 taking into consideration the fact that $e^{\D_f}(f)$ evaluated at a point $v\in V$ is the same as evaluating $f$ at $v$, the expression \eqref{eq:dComposite} equals \eqref{eq:kompproof1} by Faà di Bruno's formula.
   $$\sum\limits_{\lambda(n)}\prod\limits_{k\cdot l\in\lambda}\Big(\frac{\D_f\D_g^l(g(v))}{l!}\Big)^k\frac{n!}{k!}\Big(\Big(e^{\D_f}\Big)f(g(v))\Big)$$
   $$\implies$$
   \begin{equation}\label{eq:dCompositePoint}
   \sum\limits_{\lambda(n)}\prod\limits_{k\cdot l\in\lambda}\Big(\frac{\D_f\D_g^l(g(v))}{l!}\Big)^k\frac{n!}{k!}\Big(f(g(v))\Big)
   \end{equation}
 \end{proof}       
 The Theorem $\ref{izr:kompo}$ enables an invariant implementation of the operator of program composition in $\dP_n$, expressed as a tensor series through expressions $\eqref{eq:kompo}$ and $\eqref{eq:dComposite}$. 
 
 By fixing one mapping in  
 \begin{equation}\label{eq:opGenKompo}
\exp(\D_fe^{h\D_g}): \dP\times\dP\to\dP_\infty,
 \end{equation}
 the operator $\exp(\D_fe^{h\D_g})(g)$ performs a pullback of an arbitrary map through $g$. 
  \begin{equation}\label{eq:opKompo}
  \exp(\D_fe^{h\D_g})(g): \dP\to\dP_\infty(g)
  \end{equation}
 When projecting onto the space spanned by $\{1,\D\}$, the action of the operator $\eqref{eq:opKompo}$ is equivalent to forward mode automatic differentiation. Both forward and reverse mode are obtainable using the same operator $\eqref{eq:opGenKompo}$, by fixing the appropriate one of the two maps. This generalizes both concepts under a single operator in the theory. The expression \eqref{eq:kompo} can be generalized for the notion of a pullback, to arbitrary operators.
 
 Thus, through $\eqref{eq:kompo}$ and all its' descendants (exponents), the operator $(\ref{eq:opKompo})$ grants invariance to the point in execution of a program, which proves useful as invariants are at the center of proving algorithms' correctness. This is analogous to the principle of general covariance \cite{GeneralCovariance}[See section 7.1] in general relativity, the invariance of the form of physical laws under arbitrary differentiable coordinate transformations. This principle and what it offers to programming is explored in Section \ref{sec:TransInPractice}.
 
 With this we turn towards easing such calculations, towards completing them on the level of operators. The derivative $\frac{d}{dh}$ of $\eqref{eq:opKompo}$ is
 \begin{equation}\label{eq:dexp}
 \frac{d}{dh}\exp(\D_fe^{h\D_g})(g)=\D_f(\D_gg)e^{h\D_g}\exp(\D_fe^{h\D_g})(g)
 \end{equation}
 
 We note an important distinction to the operator $e^{h\D_g}$, the derivative of which is
 \begin{equation}\label{eq:de}
\frac{d}{dh}e^{h\D_g}=\D_ge^{h\D_g}
 \end{equation}
 We may now compute derivatives (of arbitrary order) of the pullback operator. As an example we compute the second derivative.
 $$(\frac{d}{dh})^2\exp(\D_fe^{h\D_g})(g)=\frac{d}{dh}\left(\D_f(\D_gg)e^{h\D_g}\exp(\D_fe^{h\D_g})(g)\right)$$
 which is by equations $\eqref{eq:dexp}$ and $\eqref{eq:de}$
 $$(\D_g\D_f(\D_gg))e^{h\D_g}\exp(\D_fe^{h\D_g})(g)+(\D_f(\D_gg)\D_f(\D_gg))e^{2h\D_g}\exp(\D_fe^{h\D_g})(g)$$
 using algebra and correct applications
 \begin{equation}\label{eq:d^2comp}
 (\D_f(\D^2_gg))e^{h\D_g}\exp(\D_fe^{h\D_g})(g)+(\D^2_f(\D_gg)^2)e^{2h\D_g}\exp(\D_fe^{h\D_g})(g)
 \end{equation}
 The operator is always shifted to the evaluating point $\eqref{eq:specProg}$ $v\in V$, thus, only the behavior in the limit as $h\to 0$ is of importance. Taking this limit in the expression $\eqref{eq:d^2comp}$ we obtain
 \begin{equation}
	(\D_f(\D^2_gg)+\D^2_f(\D_gg)^2)\exp(\D_f)
 \end{equation}
 
 Thus, without imposing any additional rules, we computed the operator of the second derivative of composition with $g$, directly on the level of operators. The result of course matches the equation $\eqref{eq:dComposite}$ for $n=2$.
 
 As it is evident from the example, calculations using operators are far simpler, than direct manipulations of functional series, as it was done in the proof of Theorem $\ref{izr:kompo}$. This of course enables a simpler implementation, that functions over arbitrary programming (function) spaces. In the space that is spanned by $\{\D^n\dP_0\}$ over $K$, derivatives of compositions may be expressed using only the product rule of Theorem $\ref{izr:prod}$, and $\eqref{eq:dexp}$ and $\eqref{eq:de}$, solely through the operators. Thus, explicit knowledge of rules for differentiating compositions is unnecessary, as it is contained in the structure of the operator $exp(\D_fe^{h\D_g})$ itself, which is differentiated using standard rules, as in the above example.
 
 \begin{equation}\label{eq:dkompo}
 \D^n(f\circ g)=\lim\limits_{\lVert h\rVert\to 0}(\frac{d}{dh})^n\exp(\D_fe^{h\D_g})(g,f)
 \end{equation}
 
 \begin{opomba}
 When applying $\D$ to an expression such as $(\D^n g)^k$, general rules of tensor calculus apply. 
% As the operators $\D^n$ are symmetric, this means
% \begin{equation}
% \D(\D g)^k=n(\D g)^{k-1}\D^2g
% \end{equation}
 \end{opomba}
 
   \begin{corollary}\label{izr:komp_homo}
   The operator $e^{h\D}$ commutes with composition over $\dP$
   \begin{equation}
   e^{h\D}(p_2\circ p_1)=e^{h\D}(p_2)\circ e^{h\D}(p_1)
   \end{equation}
   \end{corollary}
   
   \begin{proof}
   Follows from $\eqref{eq:pToPol}$ and Theorem $\ref{izr:kompo}$.
   \end{proof}
 
 It is useful to be able to use the $k$-th derivative of a program $P\in\dP$ as part of a different differentiable program $P_1$. As such, we must be able to treat the derivative itself as a differentiable program $P^{\prime k}\in\dP$, while only coding the original program $P$. 
\begin{trditev}\label{izr:reductionMap}
There exists a reduction of order map $\phi_n:\dP_n\to \dP_{n-1}$, such that the
following  diagram commutes
\begin{equation}\label{eq:reductionMap}
\begin{tikzcd}
  \dP_n \arrow{r}{\phi_n} \arrow{d}{\D} & 
  \dP_{n-1} \arrow{d}{\D}\\
  \dP_{n+1} 
  \arrow{r}{\phi_{n+1}}& \dP_{n}
\end{tikzcd}
\end{equation}
satisfying
\begin{equation}
\forall_{P_1\in\dP_0}\exists_{P_2\in\dP_0}\Big(\phi\circ e^\D_n(P_1)=e^\D_{n-1}(P_2)\Big)
\end{equation}
for each $n\ge 1$, where $e^\D_n$ is the projection of the operator $e^\D$ onto the set $\{\D^n\}$.
\end{trditev}      
 Thus, we gained the ability of writing a differentiable program acting on derivatives of another program, stressed as crucial (but lacking in most models) by other authors \cite{AD1}. Usage of the reduction of order map is demonstrated in Section \ref{sec:Analysis}.
 
   \subsection{Functional transformation of programs}\label{sec:FTP}
   
   We motivate the following with an example. Lets suppose a hardware $H$,
   optimized for the set of functions $F=\{f_i:V\to V\}$. The set $F$ is
   specified by the manufacturer.  
   
   With technological advances, switching the hardware is common, which can lead to a decline in performance. Thus, we would like to employ transformations of a program $P\in\dP$ in basis $F$. It is common to settle for a suboptimal algorithm, that is efficient on hardware $H$. Sub-optimality of the algorithm depends on the set $F$, whether it spans $P$, or not. A classic example of a transformation, is the Fourier transform in the basis $\{\sin(nx), \cos(mx)\}$, that spans $\dP$ (which as stated is not true for an arbitrary set $F$).
   
   Using the developed tools, the problem is solvable using linear algebra. Let
   $e^\D_n$ denote the projection of the operator $e^\D$, onto the first $n$
   basis vectors $\{\D^i\}$. We can, by Theorem $\ref{izr:e^d}$, construct a map
   $\eqref{eq:pToPol}$ from the space of programs, to the space of polynomials,
   with unknowns in $V^k$, using the operator $e^\D$. Let $\X=\{p_i\}$ denote a basis of
   the space of polynomials $V\to V$ \footnote{one choice would be the monomial basis,
   consisnting of elemets $\e_i\otimes\prod\limits_{\alpha,\forall_j}
   x_{\alpha_j}$, where $\e_i$ span $V$, $x_i$ span $V^*$ and $\alpha$
   multiindex}. Than, we can interpret $e^\D_n(P\in\dP)$ as a vector of linear combinations of $\X$, which is assumed heron.
  
  Thus, we define the tensor of basis transformation $F\to\X$
  
  \begin{equation}\label{eq:matTransF}
  T_{\X F}=
  p_1\otimes e_n^\D(f_1)^* + p_2\otimes e_n^\D(f_2)^* + \ldots + p_n\otimes e_n^\D(f_n)^*
  \end{equation}
  
  \begin{opomba}
  The tensor $\eqref{eq:matTransF}$ is a matrix, if the concerning mappings are $V^m\to V$, and tensor of rank $3$, if the concerning mappings are $V^m\to V^k$. Which can be interpreted as a sequence of matrices, one for each component of the image.
  \end{opomba}
  
  Thus, the tensor of basis transformation $\X\to F$ is
  
  \begin{equation}\label{eq:matTrans}
  T_{F\X}=T_{\X F}^{-1}
  \end{equation}
  
  Using the tensor $\eqref{eq:matTrans}$, we can easily perform basis transformations $\X\to F$. For a specific set $F$ (and consequentially a hardware $H$, upon which the set $F$ is conditioned), the tensor $\eqref{eq:matTrans}$ only has to be computed once, and can then be used for transforming arbitrary programs (while using the same operator $e^\D_n$).
  Thus, the coordinates of program $P\in\dP$ in basis $F$ are
  
  \begin{equation}\label{eq:P_F}
  	P_F=T_{F\X}\cdot e^\D(P)
  \end{equation}
  
  The expression $\eqref{eq:P_F}$ represents coordinates of program $P$ in basis $F$. Thus, the program is expressible as a linear combination of $f_i$, with components $P_F$ as coefficients.
  \begin{equation}
  P=\sum\limits_{i=0}^{n}{P_F}_if_i
  \end{equation}
  
  If $F$ does not span $\dP$, or we used the projection of the operator $e^\D_{n<N}$, the expression $P_F$ still represents the best possible approximation of the original program, on components $\{\D^n\}$, in basis $F$.
  
  It makes sense to, before computing the tensor $\eqref{eq:matTrans}$, to expand the set $F$, by mutual (nested) compositions, and gain mappings, that can not be expressed as linear combinations (but are still optimized for hardware $H$), and so increasing the power of the method.
\subsection{The special case of functions $\VV\to K$}
We describe a special case when $\dP_0=\VV\otimes\tilde{\dP}_0$, where
$\tilde{\dP}_{-1}<K^\VV$ is a subspace of the space of functions $\VV\to K$. This is
useful, if the set of functions $F$ only contains functions $\VV\to K$. It is
very common, that basic operations in a programming language change one single
real valued variable at a time. In that case, the value of changed variable is
described by the function $\tilde{f}:\VV\to K$, while the location, where the
value is saved is given by a standard basis vector $\e_i$. The map $f:\VV\to \VV$
is then given as a tensor product $f=\e_i\otimes \tilde{f}$. We can start our
construction of programming spaces by defining differentiable programing space
of functions $\VV\to K$ instead of maps $\VV\to \VV$ as in definition
\ref{def:dP}. Analog to the  the theorems \ref{izr:P}, \ref{izr:P_n} it is easy
to verify, that 
\begin{equation}
  \label{eq:tilda_dP}
  \D^k\tilde{\dP_0}\subset\tilde{\dP_0}\otimes T_k(\VV^*).
\end{equation}

 \subsection{Control structures}
 
 Until now, we restricted ourselves to operations, that change the memories' content. Along side assignment statements, we know control statements (ex. statements \texttt{if},
  \texttt{for}, \texttt{while}, ...). Control statements don't directly influence values of variables, but change the execution tree of the program. This of course affects the derivative. But, for a certain set of input variables, the execution of the program will always be the same. This is why we interpret control structures as a definition of a spline.
  
 Each control structure divides the space of parameters into different domains, in which the execution of the program is always the same. The entire program divides the space of all possible parameters to a finite set of domains $\{\Omega_i;\quad i=1,\ldots
  k\}$, where the programs' execution is always the same. As such, a program may in general be defined as a spline. For $\vec{x}\in\RR^n$
 \begin{equation}
   \label{eq:zlrprk_splosno}
   P(\vec{x}) =
   \begin{cases}
     P_{n_11}\circ P_{(n_1-1)1}\circ\ldots P_{11}(\vec{x});&\quad \vec{x}\in\Omega_1\\
     P_{n_22}\circ P_{(n_2-1)2}\circ\ldots P_{12}(\vec{x});&\quad \vec{x}\in\Omega_2\\
     \vdots&\quad\vdots\\
     P_{n_kk}\circ P_{(n_k-1)k}\circ\ldots P_{1k}(\vec{x});&\quad \vec{x}\in\Omega_k\\
   \end{cases}
 \end{equation}
 The operator $e^\D$ (at some point) of a program $P$, is of course dependent on initial parameters $\vec{x}$, and can also be expressed as a spline, inside domains $\Omega_i$
 \begin{equation}
   \label{eq:Dzlrprk_splosno}
   e^\D P({\vec{x}}) =
   \begin{cases}
     e^\D P_{n_11}\circ e^\D P_{(n_1-1)1}\circ\ldots\circ e^\D P_{11}(\vec{x});&\quad \vec{x}\in\interior(\Omega_1)\\
     e^\D P_{n_22}\circ e^\D P_{(n_2-1)2}\circ\ldots\circ e^\D P_{12}(\vec{x});&\quad \vec{x}\in\interior(\Omega_2)\\
     \vdots&\quad\vdots\\
     e^\D P_{n_kk}\circ e^\D P_{(n_k-1)k}\circ\ldots\circ e^\D P_{1k}(\vec{x});&\quad \vec{x}\in\interior(\Omega_k)\\
   \end{cases}
 \end{equation}

 Problems with differentiability which might have occurred on the edge of domains $\partial\Omega_i$, are removed from the realm of possibilities by the way our theory has been constructed.
 
 \begin{trditev}
 Each domain $\Omega_i$ can be treated as an open set.
 \end{trditev}
 \begin{proof}
 Each execution of program \eqref{eq:zlrprk_splosno} is computed on the interior one of the domains $\Omega_i$. The same holds true for all coefficients in the tensor series  \eqref{eq:Dzlrprk_splosno}, as they are computed on the same point in the interior of the domain $\Omega_i$ (as opposed to numeric differentiation, where two points would be employed, which may each belong the interior of a different domain). Therefore, all computations are computed on the interior of some domain $\Omega_i$, effectively making it an open set.
 \end{proof}

\subsection{Transformations in practice} \label{sec:TransInPractice}

Branching of programs into domains $\eqref{eq:zlrprk_splosno}$ is done through conditional statements. Each conditional causes a branching in programs' execution tree.

\begin{trditev}\label{izr:st.zlepkov}
Cardinality of the set of domains $\Omega=\{\Omega_i\}$ equals $\lvert\{\Omega_i \}\rvert=2^k$, where $k$ is the number of conditionals contained within the program.
\end{trditev}
\begin{opomba}
Iterators, that do not change exit conditions within its' body, do not cause branching.
\end{opomba}

It follows from Claim $\ref{izr:st.zlepkov}$, that the complexity of naive implementations of methods presented in Sections $\ref{sec:Vrsta}$ and $\ref{sec:FTP}$ are exponential (if we were to treat each domain $\Omega_i$ by itself). But, with correct application of Theorems developed in this paper, we may drastically reduce this down to linear. This is the subject of study in this section.
\vspace{10px}

\tikzstyle{decision} = [diamond, draw, fill=blue!20, 
    text width=4.5em, text badly centered, node distance=3cm, inner sep=0pt]
\tikzstyle{block} = [rectangle, draw, fill=blue!20, 
    text width=5.5em, text centered, rounded corners, minimum height=4em]
\tikzstyle{line} = [draw, -latex']

\begin{figure}[!h]
\centering
\begin{tikzpicture}[node distance = 2cm, auto]
    \node [block] (p1) {$P_1$};
    \node [block,right of=p1,node distance=5cm] (ep1) {$e^\D(P_1)$};
    \node [block,right of=ep1,node distance=5cm] (tep1) {$T_{F\X}\cdot e^\D(P_1)$};
    
    \node [decision, below of=ep1,node distance=3cm] (v1) {\small{Branching}};
    
    \node [block,below of=v1,node distance=2.75cm] (ep2) {$e^\D(P_2)$};
    \node [block,left of=ep2,node distance=3.5cm] (p2) {$P_2$};
    \node [block,right of=ep2,node distance=3.5cm] (tep2) {$T_{F\X}\cdot e^\D(P_2)$};
    
    
    \node [block,below of=ep2,node distance=2.75cm] (ep3) {$e^\D(P_3)$};
    \node [block,left of=ep3,node distance=5cm] (p3) {$P_3$};
    \node [block,right of=ep3,node distance=5cm] (tep3) {$T_{F\X}\cdot e^\D(P_3)$};
    
    \path [line,dashed] (p1) -- node{$e^\D$}(ep1);
    \path [line,dashed] (p2) -- node{$e^\D$}(ep2);
    \path [line,dashed] (p3) -- node{$e^\D$}(ep3);
   
    \path [line,dashed] (ep1) -- node{$T_{F\X}$}(tep1);
    \path [line,dashed] (ep2) -- node{$T_{F\X}$}(tep2);
    \path [line,dashed] (ep3) -- node{$T_{F\X}$}(tep3);
		
	\coordinate [below of=ep2,node distance=1.5cm](C);
	\coordinate [right of=C,node distance=5.25cm](D);
	\coordinate [right of=v1,node distance=5.25cm](E);
	\coordinate [above of=E,node distance=1.5cm](F);
	\coordinate [above of=v1,node distance=1.5cm](G);
	\path[line](C)--(D)--(E)--(F)--(G);
	
	\coordinate [above of=ep2,node distance=1.5cm](C2);
	\coordinate [above of=p2,node distance=1.5cm](D2);
	\coordinate [above of=tep2,node distance=1.5cm](E2);
	\path[line](v1)--(ep2);
	\path[line] (C2)--(D2)--(p2);
	\path[line] (C2)--(E2)--(tep2);
	
	\coordinate [above of=ep3,node distance=1cm](D3);
	\coordinate [below of=ep2,node distance=1cm](C3);
	\coordinate [below of=p2,node distance=1cm](D3);
	\coordinate [below of=tep2,node distance=1cm](E3);
	\coordinate [below of=C,node distance=0.25cm](CC3);
	\draw (p2)--(D3)--(E3)--(tep2);
	\draw (ep2)--(C);
	
	\coordinate [below of=ep1,node distance=1cm](C4);
	\coordinate [below of=p1,node distance=1cm](D4);
	\coordinate [below of=tep1,node distance=1cm](E4);
	\draw (p1)--(D4)--(C4);
	\draw (tep1)--(E4)--(C4);
	\path[line](ep1)--(v1);
	
	\coordinate [above of=p3,node distance=1.25cm](C5);
	\coordinate [above of=ep3,node distance=1.25cm](D5);
	\coordinate [above of=tep3,node distance=1.25cm](E5);

	
	\coordinate [left of=p3,node distance=1.75cm](D2);
	\coordinate [left of=v1,node distance=6.75cm](F2);
	\coordinate [below of=D2,node distance=1.25cm](H2);
	\coordinate [below of=p3,node distance=1.25cm](I2);
	\coordinate [below of=ep3,node distance=1.25cm](J2);
	\coordinate [below of=tep3,node distance=1.25cm](K2);
	
	\path[line](v1)--(F2)--(D2)--(H2)--(I2)--(p3);
	\path[line](I2)--(J2)--(ep3);
	\path[line](J2)--(K2)--(tep3);
	
\end{tikzpicture}
\vspace{3px}
\caption{Transformation diagram} \label{fig:diagram} 
\end{figure}

\begin{izrek}
A program $P\in\dP$ can be equivalently represented with at most $2n+1$ applications of the operator $e^\D$, on $2n+1$ analytic programs.
\end{izrek}

\begin{proof}
	Source code of a program $P\in\dP$ can be represented by a directed graph, as shown in Figure $\ref{fig:diagram}$. Each branching causes a split in the execution tree, that after completion returns to the splitting point.
	By Theorem $\ref{izr:kompo}$, each of these branches can be viewed as a program $p_i$, for which it holds $$e^\D(p_n\circ p_{n-1}\circ\cdots\circ p_1)=e^\D(p_n)\circ e^\D(p_{n-1})\circ\cdots\circ e^\D(p_1)$$ by Theorem $\ref{izr:kompo}$.
	
	Thus, the source code contains $2n$ differentiable branches, from its' first branching on, not counting the branch leading up to it, upon which the application of the operator $e^\D$ is needed. Total of $2n+1$. By Theorem $\ref{izr:P_n}$, each of these branches is analytic.
\end{proof}

Images of the operator $e^\D$ and $T_{F\X}$  are elements of the original space $\dP$, which may be composed. Thus, for $P=p_3\circ p_2\circ p_1$, the following makes sense
\begin{equation}
P=\Big(p_3\circ e^ \D(p_2)\circ T_{F\X}e^\D(p_1)\Big) \in \dP
\end{equation}

The same holds true for all permutations of applications of operators $e^\D$, $T_{F\X}$ and $id$, as visible in Figure $\ref{fig:diagram}$.

\begin{opomba}
In practice, we always use projections of the operator $e^\D$ to some finite subset $\DD^n$, resulting in $e^\D_n$. Therefore, we must take note that the following relation holds
\begin{equation}
e^\D_m(P_2)\circ e^\D_n(P_1)=e^\D_k(P_2\circ P_1)\iff 0\le k\le \min(m,n)
\end{equation}
when composing two images of the applied operator, projected to different subspaces.
\end{opomba}

The transformation tensor $T_{F\X}$ is needed to be computed only once and can then on be applied to any program running on said hardware. The same holds true for each branch $p_i$, which can, by Theorem $\ref{izr:komp_homo}$, be freely composed amongst each other.

\begin{trditev}
Images of the operator $e^\D (P\in\dP_0)$ are elements of $U\otimes S(V^*)$ by $\eqref{eq:specProg}$, consisting of multi-linear maps. As such, their evaluation and composition ($e^\D(P_1)\circ e^\D(P_2)$) is tailor made for efficient implementation by methods of parallelism \cite{TensorGPU}, with computable complexities.
\end{trditev}

Transformations $\eqref{eq:P_F}$ need to be computed only once. Complexity of the composition of images of transformations $T_{F\X}\cdot e^\D(p_2)\circ T_{F\X}\cdot e^\D(p_1)$ depends on complexities of $f_i\in F$.

Thus, we have acquired a tool, enabling us to choose the complexity of arbitrary algorithms $P\in\dP$, along with estimations and order of the error of the approximation. Simultaneously it enables transformations of the function basis of programs, and with it implementations, more inclined towards efficiency on a given hardware. After initial applications of operators and performed transformations the program may be used, as many times as pleased, towards the purpose it was initially designed for. By doing so, we may manipulate its' complexity (for $k\le n$) and switch between different code formulations ($p$, $e^\D(p)$, $T_{F\X}\cdot e^\D(p)$), as better suited to the nature of the domains' subset, upon which the program operates at a given moment.
  
\section{Analysis}\label{sec:Analysis}
  
   The theory presented in this paper grants new insights through methods of analysis. It reveals a procedure to be a system exerting change on objects inhabiting virtual space. These revelations are the object of study of this section.
  
  \subsection{Study of properties}
  
  We will denote the fact, that some object $v$ has the property $X$, by $v\in X$. Suppose we have $v\notin X$, and desire a procedure that in some way modifies $v$ to have the property $X$. The desire is a procedure $P:v\notin X\to P(v)\in X$, that modifies $v$ in some way. Let $S$ be some relation of similarity. We have
  \begin{equation}
  p\in P:v\notin X\to P(v)\in X\iff vSp(v)
  \end{equation}
  Usually such procedures are difficult to construct and may not even explicitly exist. An easier task is to construct a procedure $T\in\dP_0$, whose output allows deduction of whether $v$ has the property $X$, or does not. We generalize such notions and develop methods of analyzing procedures $T\in\dP_0$ that enable a construction of an algorithm
  \begin{equation}\label{eq:algA}
  A:T\to P
  \end{equation}
  
  To achieve this, we first develop a way of probing the inner structure of a procedure and explore procedure $T$ and how it interacts with the elements of the domain. We employ the developed operational calculus to this effect. Using it, we devise a way of extracting conceptual steps of importance $T_i$ to a partitioning $X_j$, from a procedure $T\in\dP_0$. The sets of conceptual steps of importance to a partitioning $X_j$ is denoted by $T_{X_j}$. After extraction of $T_{X_j}$, we use it for the construction of algorithm $A$ \eqref{eq:algA}.
  
  Let $R$ be an equivalence relation 
  \begin{equation}\label{eq:equivRelation}
  v_1Rv_2\iff T(v_1)ST(v_2)
  \end{equation}
  where $S$ is some relation of similarity (ex. $\lvert1- T(v_1)\rvert\le c\land\lvert1- T(v_2)\rvert\le c$) and $M$ be the set of interest. Than $T$ partitions $M$ into equivalence classes, as by $R$ $\eqref{eq:equivRelation}$. The above property $X$, may be viewed as one of these equivalence classes $X_i$. Note that in general the relation need not be an equivalence, as it is here so solely for ease of illustration. 
  
   Let $T=T_n\circ T_{n-1}\circ\cdots\circ T_1$, for simplicity, but it could just as easily be a spline as in $\eqref{eq:zlrprk_splosno}$ (by a specific $T_i$ containing a control structure). Here $T_i=\varepsilon_k\circ\ldots \varepsilon_1$, is a conceptual step in the procedure. Then, we can inquire about and measure the importance $K$ of an arbitrary $T_i$, to a partitioning into a specific equivalence class $X_j$. $\eqref{eq:equivRelation}$. Thus, we are able to extract $T_{X_j}$, the set of steps of importance, to such a partitioning. We treat $T_i\circ\cdots T_1$ as the input of the procedure $T^n_i=T_n\circ\cdots\circ T_{i+1}$.
  
  We present a simple algorithm towards such a goal, which can easily be enhanced with a more sophisticated choice of the measure of importance, as made available by the theory. For simplicity of the illustration, we use the norm of the differential as a measure, as it measures the rate of change in the relation. A measure easily replaced by a more sophisticated one, as allowed by the theory.
  
  Extraction of the derivate of the sub-procedure $T^n_k$, with respect to the parameters of the conceptual step $T_k$, is accomplished by
  \begin{equation}
  \D_{T_k}T^n_k=proj_{\{\D\}}\left(e^{\D_{T_k}}_1(T^n_k)\right)
  \end{equation}
  applying the operator $e^{\D_{T_k}}_1$ onto the sub-procedure, resulting in $e^{\D_{T_k}}_1(T^n_k)\in\dP_1$ and projecting the result onto the subspace spanned by $\{\D\}$.
  
\begin{algorithm}[h]
\caption{Steps of importance to a partitioning}
\label{alg:partitioning}
\begin{algorithmic}[1]
\Procedure{Steps of importance to a partitioning}{}
\For{each $X_i$}
\For{each $v\in X_i$}
\For{each $T^n_k$}
\State add norm $\D_{T_k}T^n_k$ to $N_{T_k}$
\EndFor
\EndFor
\State threshold $T_t\in T_{X_i}\iff N_{T_t}\ge c$  
\State insert $T_{X_i}$ in $I$
\EndFor
\State return $I$
\EndProcedure
\end{algorithmic}
\end{algorithm}
Thus we get $I=\{T_{X_j}\}$, where $T_{X_j}$ is the set of steps $T_i$ measured as important by Algorithm $\ref{alg:partitioning}$. Each set $T_{X_j}$ contains conceptual steps, portions of the source code, whose action on the input contributes most in a partitioning to the equivalence class $X_j$. As such, the set $I$ shows to be useful in studying properties of the set $M$, as it reveals to what action, a property of an element $v\in M$ reacts to, as we probe it using the method. It may be used to better understand procedures and interpret them. The set $I=\{T_{X_j}\}$ can reveal procedural steps, that may have previously been thought of as less important to the end result.

With this we turn towards constructing algorithm $A$ $\eqref{eq:algA}$. As shown by Algorithm $\ref{alg:partitioning}$, we arrived at sub-procedures $T_i$, to which properties $X_j$ react to most strongly. Thus, using the algebra and operational calculus presented in this paper, we can construct a procedure that changes an arbitrary $v\in M$, in such a way that maximizes the action of $T_i\in T_{X_j}$, for the desired property $X_j$. Global maxima of the method can be viewed as archetypes of objects in the domain possessing a specific  property $X_j$. 
\begin{opomba}\label{rem:iPartitioning}
We assume that the set $I$ induces some partitioning on the domain, where intersections need not be empty, thus containing some information about the domain, as it is common in practice and assumed so heron.
\end{opomba}


We extract a once differentiable derivative of $T^n_j$ needed by
\begin{equation}\label{eq:T'}
T^\prime_i=\phi\circ e^{\D_{T_i}}_2(T^n_i)\in\dP_1
\end{equation}
as, its' derivative (second derivative of $T$) is needed for further compositions $\eqref{eq:kompo}$ by Theorem $\ref{izr:kompo}$, where $\phi$ is the reduction of order map $\eqref{eq:reductionMap}$ of Theorem $\ref{izr:reductionMap}$. As such, it holds $T^\prime_i\in\dP_1$. We do so, for each $T_i\in T_{X_j}\subset I$, where $X_j$ is the desired property.

Let $C\in\dP$ be a convex map, some measure of magnitude. Thus, we arrive at the expression to be maximized
\begin{equation}
E_i=C\circ T^\prime_i
\end{equation}
\begin{algorithm}[h]
\caption{Appoint property $X_j$ to $v\in M$}
\label{alg:appoint}
\begin{algorithmic}[1]
\Procedure{Appoint property $X_j$ to $v\in M$}{}
\State initialize path $\gamma$ with $v$
\For{each step}
\For{each $T_i\in T_{X_j}$}
\State extract $T^\prime_i=\phi\circ e^{\D_{T_i}}_2(T^n_i)$ $\eqref{eq:T'}$
\State compute the energy $E_i=e^{\D_v}_1(C)\circ T^\prime_i$
\State extract the derivative $\D_v E_i=proj_{\{\D\}}(E_i)$
\State add $\D_v E_i$ to $\D_v E$
\EndFor
\State update $v$ by $step(\D_v E,v)$
\State insert $v$ to $\gamma$
\EndFor
\State return $\gamma$
\EndProcedure
\end{algorithmic}
\end{algorithm}
The derivative with respect to $v$ is extracted by applying $e^{\D_{v}}_1$ onto $C$ mapping into $\dP_1$, composing it with $T^\prime_i\in\dP_1$ and projecting the image onto the subspace spanned by $\{\D\}$,
\begin{equation}
\D_vE_i=proj_{\{\D\}}(e^{\D_v}_1(C)\circ T^\prime_i)
\end{equation}
as it needs not to be differentiable for this purpose.

For simplicity, assume $step$ to represent one maximization step, using a first derivative in its' execution. A method easily enhanced by any of more sophisticated optimization techniques, as made available by the theory. 

As shown by Algorithm $\ref{alg:appoint}$, we arrived at a path $\gamma$, whose end point is the transformed $v\in M$, now endowed with the property $X_j$. Thus, using the theory, with the procedure $T$ as an input variable in Algorithm $\ref{alg:appoint}$, we constructed algorithm $A: T\to P$ $\eqref{eq:algA}$. When $v\in M$ represents an object upon which we wish to impose the property $X_j$, we may do so using the derived procedure $A(T)=P$
\begin{equation}\label{eq:procedureP}
P:v\notin X_j\to v\in X_j
\end{equation}

The derived procedures $P$ $\eqref{eq:procedureP}$, are a generalization of special cases such as Google's Deep Dream project \cite{DeepDream}, where the program represents a neural network, with neurons with high activity filling the set of important conceptual steps $T_{X_j}$ and the resulting $v\in X_j$ is an altered image.

\begin{definicija}[Path to change]
Output of Algorithm \ref{alg:appoint} is $\gamma$, the path to change.
\end{definicija}

Path to change is the flow curve outputted by Algorithm $\ref{alg:appoint}$. When $T$ serves as a simulation, with $v$ modeling a real-life object of study, the procedure opens new doors in analyzing real-life phenomena. Through it, we may observe how $v$ evolves through iterations and study different stages of change, which serves as a useful insight when designing procedures causing change in real-life phenomena.

\subsection{Computational dynamics}

As each procedure is revealed to be a system existing as an object in virtual space, its computational capability becomes synonymous with shape. For example, training of neural network is a homotopy in this space, as it converges to its optimal computational form in a high-dimensional manifold.
The developed operational calculus allows as to sculpt and mold a procedures shape, and through the established connection, its computational capability. There is a strong connection between shape of the inhabiting space and physical principles it imposes upon objects traversing it \cite{GeneralCovariance}.

As procedures are systems entered by an object as an input and exited as an output upon undergoing change, the process of computation can be viewed as evolutions in phase space. Thus, our newfound ability to mold shape in virtual space enables us to impose principles into the system the procedure represents. Assume we have some a priori knowledge of how the process is to change the object, some previously known heuristic $H$ known from practice. That is equivalent to having knowledge on the path to change $\gamma$, as argued in the previous section. Heuristic $H$ might be some measure on the environment, or a direction, for objects at a particular location $L_i$ in the system. We express change the desired heuristic $H$ locally exerts with a relation.
Thus, $H$ represents a local principle we wish to impose into the system. We denote the fact the system obeys the principle $H$ by $P\in H$. The desire is an algorithm molding the system, modifying it in some way, as to impose the principle into it.
\begin{equation}\label{eq:impose}
A:P\notin H\to P\in H
\end{equation}

To achieve this, we again probe the system represented by procedure $P\in\dP_0$. By knowing from practice, or otherwise, at which locations (inputs) of the system, the effects of heuristic $H_i$ should be present, we maximize its action in the neighborhood of $L_i$. This effectively means that objects at locations $L_i$ in the system are affected by the principle $H_i$.
Let
\begin{equation}
P=P_n\circ P_{n-1}\circ\cdots\circ P_1
\end{equation}
be the procedure representing the system for simplicity, but it could just as easily be a spline as in $\eqref{eq:zlrprk_splosno}$ (by a specific $P_i$ containing a control structure), with $P_i=p_k\circ\cdots\circ p_1$ being the conceptual step treated by heuristic $H_j$. $L_j$ are the concerning locations. Again, we let $P^i_k=P_i\circ\cdots\circ P_k$. Directions are usually modeled by first derivatives, and attractions by second derivatives, so lets assume employment of both by the heuristic $H$. For simplicity, assume $step$ to represent one optimization step, using a first derivative in its' execution. A method easily enhanced by any of more sophisticated optimization techniques as made available by the theory.

As $step$ requires the derivative of $H$, which itself contains first and second derivatives of $P$, requires use of the operator $e^\D_3$, for the second derivative to be differentiable. Once differentiable first and second derivatives with respect to parameters $p$ are extracted by
\begin{equation}
P^{i\prime}_1=proj_{\{1,\D\}}\left(\phi\circ e^{\D_{p}}_3(P^i_1)\right)\in\dP_1
\end{equation}
\begin{equation}
P^{i\prime\prime}_1=\phi^2\circ e^{\D_p}_3(P_i)\in\dP_1
\end{equation}
As both are once differentiable, so is the heuristic $H$ taking them as input.
\begin{equation}
H^\prime=e^{\D_p}_1(H)\circ(P^{i\prime}_1,P^{i\prime\prime}_1)\in\dP_1
\end{equation}
Its derivative is extracted by projecting $H^\prime$ onto $\{\D\}$, as it need not be further differentiable for the use in the $step$ function.

\begin{algorithm}[h]
\caption{Imposing a principle $H$}
\label{alg:impose}
\begin{algorithmic}[1]
\Procedure{Imposing a principle $H$}{}
\For{each step}
\For{each $L_j\in L$}
\State extract $P^{i\prime}_1=proj_{\{1,\D\}}\left(\phi\circ e^{\D_{p}}_3(P^i_1)\right)$
\State extract $P^{i\prime\prime}_1=\phi^2\circ e^{\D_p}_3(P_i)$
\State extract $H^\prime=e^{\D_p}_1(H)\circ(P^{i\prime}_1,P^{i\prime\prime}_1)$
\State add $proj_{\{\D\}}(H^\prime_j)$ to $H^\prime$
\EndFor
\State update $p$ by $step(H,p)$
\EndFor
\EndProcedure
\end{algorithmic}
\end{algorithm}

As shown by Algorithm \ref{alg:impose}, we arrived at a procedure $P$, whose conceptual step $P_i$ maximizes action of the principle $H$ upon objects inhabiting it as a system. Using the theory, with the system $P$ as an input variable in Algorithm \ref{alg:impose}, we constructed the algorithm $A:P\notin H\to P\in H$ $\eqref{eq:impose}$.

As we traverse a system looking for a solution, we learn about it in the process. Exploring the virtual space, we may establish connections with existing heuristics, or discover new ones and immediately alter the way the search is conducted. As demonstrated, we are capable of not only altering properties of object in the domain, but also altering the system itself. We employ this to impose principles into it to best serve our quest for a solution, effectively rendering the problem more inclined towards being solved.


\section{Conclusions}

Existence of a procedure is embedded in a virtual reality, forming a system of objects undergoing change in a virtual space. Just as the reality inhabited by us has been studied, revealing principles and laws, so can the virtual reality inhabited by procedures. Yet here lies a tougher task, as the laws of the system changing an object to a desired state are simultaneously observed and constructed. This reinforces the need for a language capable of not only capturing, but also constructing digital phenomena as actions on a virtual space, a feat demonstrated by the theory presented in this paper. 

Like Feynman \cite{Feynman} and Heaviside \cite{Operational} for physics before us, we developed operational calculus on programming spaces, allowing analytic conclusions through algebraic means, easing implementation. We derived the operator of program composition, generalizing both forward and reverse mode of automatic differentiation (of arbitrary order), under a single operator in the theory. Both the use of algebra and calculus were demonstrated upon the operator, as calculations and manipulations were performed on the operator level, before the operator is applied to a particular program.
This language condenses complex notions into simple expressions, enabling formulation of meaningful algebraic equations to be solved. Doing so, we derived functional transformations of programs in arbitrary function basis', a useful tool when adapting code to the specifics of a given hardware, among others.
All such formulations are invariant not only to the choice of a programming space, but also to the point in execution of a program, introducing the principle of general covariance \cite{GeneralCovariance} to programming. Offerings of this principle were exploited, as we designed methods on how transformations are to be interchangeably applied in practice. These methods allow for seamless transitions between transformed forms and original code throughout the program. We showed how runtime complexity of the transformed form is computed, gaining the ability to choose the programs complexity, as we tailor the employed form to the specifics of the domains' subset being operated on at a given moment. Thus, the designing process retained and expanded on the freedom of algorithmic expression programming is embraced for.


  \printbibliography
\end{document}

